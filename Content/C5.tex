\newpage

\section[Day 5: Line Integrals]{ Line Integrals }

\subsection{ Parametrized Curves }

    \begin{definition}{Path}{14cm}
        A {\color{lblue} path} in $\mathbb{R}^n$ is a continuous
        C(t): [a,b] $\rightarrow$ $\mathbb{R}^n$.

        \vspace{0.3cm}
        If C(t) is twice-differentiable, then the
        {\color{lblue} velocity} of C(t) at $t_0$ $\in$ [a,b]:

        \hspace{0.5cm}
        v($t_0$) = x'($t_0$)
        
        Also, the aceleration of C(t) at $t_0$ $\in$ [a,b]:

        \hspace{0.5cm}
        a($t_0$) = x''($t_0$)
    \end{definition}

    \vspace{0.5cm}



    \begin{definition}{Arclength}{14cm}
        The length of a $C^1$ path C(t): [a,b] $\rightarrow$ $\mathbb{R}^n$:

        \hspace{0.5cm}
        L = $\int_a^b$ $||C'(t)||$ dt
    \end{definition}

    \begin{proof}
        Choose \{$x_1,...,x_n$\} $\in$ [a,b] such that each $x_i$ $<$ $x_{i+1}$.
        Then the length of C(t):

        \hspace{0.5cm}
        L = $\lim_{n \rightarrow \infty}$ $\sum_{i=1}^n$ $||C(x_i) - C(x_{i-1})||$

        Let each C($x_i$) = $(C_1(x_i),...,C_n(x_i))$. Thus:
        
        \hspace{0.5cm}
        $||C(x_i) - C(x_{i-1})||$
        = $\sqrt{(C_1(x_i)-C_1(x_{i-1}))^2 + ... + (C_n(x_i)-C_n(x_{i-1}))^2}$

        Since C(t) is $C^1$, by the Mean Value Theorem,
        there is a $t_{i_k}$ $\in$ $[x_{i+1},x_i]$ such that:

        \hspace{0.5cm}
        $C_k(x_i) - C_k(x_{i-1})$
        = $(x_i - x_{i-1})C_1'(t_{i_k})$

        Thus:

        \hspace{0.5cm}
        $||C(x_i) - C(x_{i-1})||$
        = $\sqrt{(x_i-x_{i-1})^2[C_1'(t_{i_1})]^2
                + ... + (x_i-x_{i-1})^2[C_1'(t_{i_n})]^2}$

        \hspace{0.5cm}
        L = $\lim_{n \rightarrow \infty}$ $\sum_{i=1}^n$
            $\sqrt{[C_1'(t_{i_1})]^2 + ... + [C_1'(t_{i_n})]^2}(x_i - x_{i-1})$
        
        \hspace{0.9cm}
        = $\int_a^b$ $\sqrt{[C_1'(t)]^2 + ... + [C_n'(t)]^2}$ dt
        = $\int_a^b$ $||C'(t)||$ dt 
    \end{proof}

    \vspace{0.5cm}



    \begin{example}
        Find the length of the helix C(t) = (5cos(t),5sin(t),12t)
        for t $\in$ $[0,2\pi]$.
    \end{example}

    \begin{tbox}
        C'(t) = (-5sin(5),5cos(t),12)
        so $||C'(t)||$ = $\sqrt{25\sin^2(t) + 25\cos^2(t) + 144}$ = 13.

        L = $\int_0^{2\pi}$ 13 dt = $26\pi$
    \end{tbox}

    \vspace{0.5cm}





\subsection{ Vector Fields }

    \begin{definition}{Vector Field and Flow Lines}{14cm}
        A {\color{lblue} vector field} on $\mathbb{R}^n$ is
        F: X $\subset$ $\mathbb{R}^n$ $\rightarrow$ $\mathbb{R}^n$

        \vspace{0.5cm}

        A {\color{lblue} flow line} of vector field F
        is a differentiable path C(t): [a,b] $\rightarrow$ $\mathbb{R}^n$
        such that:

        \hspace{0.5cm}
        C'(t) = F(C(t))
    \end{definition}

    \vspace{0.5cm}



    \begin{definition}{Del Operator}{14cm}
        The {\color{lblue} Del Operator} on $\mathbb{R}^n$:
        
        \hspace{0.5cm}
        $\nabla$ = $\sum_{i=1}^n$ $\frac{\partial}{\partial x_i}e_i$
        = $(\frac{\partial}{\partial x_1},...,\frac{\partial}{\partial x_n})$
    \end{definition}

    \newpage



    \begin{definition}{Divergence}{14cm}
        For differentiable vector field F: X $\subset$ $\mathbb{R}^n$
        $\rightarrow$ $\mathbb{R}^n$, let F = $(F_1,...,F_n)$.

        Then the {\color{lblue} divergence}, div:
        $\mathbb{R}^n$ $\rightarrow$ $\mathbb{R}$, of F:

        \hspace{0.5cm}
        div(F)
        = $\nabla$ $\cdot$ F
        = $\frac{\partial F_1}{\partial x_1}
                + ... + \frac{\partial F_n}{\partial x_n}$

        If div(F) = 0 everywhere, then F is incompressible.
    \end{definition}

    \vspace{0.5cm}



    \begin{definition}{Curl}{14cm}
        For differentiable vector field F: X $\subset$ $\mathbb{R}^3$
        $\rightarrow$ $\mathbb{R}^3$, let F = $(F_1,F_2,F_3)$.

        Then the {\color{lblue} curl}, curl:
        $\mathbb{R}^n$ $\rightarrow$ $\mathbb{R}^n$, of F: 

        \hspace{0.5cm}
        curl(F)
        = $\nabla$ $\times$ F =
        $
        \begin{bmatrix}
            e_1 & e_2 & e_3 \\
            \frac{\partial}{\partial x} & \frac{\partial}{\partial y}
                & \frac{\partial}{\partial z} \\
            F_1 & F_2 & F_3
        \end{bmatrix}
        $
        = ($\frac{\partial F_3}{\partial y}-\frac{\partial F_2}{\partial z} \ , \
            \frac{\partial F_1}{\partial z}-\frac{\partial F_3}{\partial x} \ , \
            \frac{\partial F_2}{\partial x}-\frac{\partial F_1}{\partial y}$)

        If curl(F) = 0 everywhere, then F is irrotational.
    \end{definition}

    \vspace{0.5cm}



    \begin{wtheorem}{Vector fields from Gradients are irrotational}{14cm}
        If f: X $\subset$ $\mathbb{R}^3$ $\rightarrow$ $\mathbb{R}$
        is $C^2$, then curl($\nabla f$) = 0
    \end{wtheorem}

    \begin{proof}
        Since $\nabla f$
        = $(\frac{\partial f}{\partial x},
            \frac{\partial f}{\partial y},
            \frac{\partial f}{\partial z})$,
        then by {\color{red} theorem 2.3.9}:

        \hspace{0.5cm}
        curl($\nabla f$) =
        $
        \begin{bmatrix}
            e_1 & e_2 & e_3 \\
            \frac{\partial}{\partial x} & \frac{\partial}{\partial y}
                & \frac{\partial}{\partial z} \\
            \frac{\partial f}{\partial x} & \frac{\partial f}{\partial y}
                & \frac{\partial f}{\partial z}
        \end{bmatrix}
        $
        = ($\frac{\partial^2 f}{\partial y\partial z}
                - \frac{\partial^2 f}{\partial z\partial y} \ , \
            \frac{\partial^2 f}{\partial z\partial x}
                - \frac{\partial^2 f}{\partial x\partial z} \ , \
            \frac{\partial^2 f}{\partial x\partial y}
                - \frac{\partial^2 f}{\partial y\partial x}$)
        = 0
    \end{proof}

    \vspace{0.5cm}



    \begin{wtheorem}{The curl is incompressible}{14cm}
        If F: X $\subset$ $\mathbb{R}^3$ $\rightarrow$ $\mathbb{R}^3$
        is $C^2$, then div(curl(F)) = 0
    \end{wtheorem}

    \begin{proof}
        Since curl(F)
        = ($\frac{\partial F_3}{\partial y}-\frac{\partial F_2}{\partial z} \ , \
            \frac{\partial F_1}{\partial z}-\frac{\partial F_3}{\partial x} \ , \
            \frac{\partial F_2}{\partial x}-\frac{\partial F_1}{\partial y}$).
        then by {\color{red} theorem 2.3.9}:

        \hspace{0.5cm}
        div(curl(F))
        = $\frac{\partial}{\partial x}
            (\frac{\partial F_3}{\partial y}-\frac{\partial F_2}{\partial z})$
            + $\frac{\partial}{\partial y}
            (\frac{\partial F_1}{\partial z}-\frac{\partial F_3}{\partial x})$
            + $\frac{\partial}{\partial z}
            (\frac{\partial F_2}{\partial x}-\frac{\partial F_1}{\partial y})$
        = 0
    \end{proof}

    \vspace{0.5cm}





\subsection{ Line Integrals }

    \begin{wtheorem}{Scalar Line Integral}{14cm}
        Let $C^1$ path C(t): [a,b] $\rightarrow$ $\mathbb{R}^n$
        and f: C([a,b]) $\subset$ X $\subset$ $\mathbb{R}^n$ $\rightarrow$
        $\mathbb{R}$ be continuous.

        Then the {\color{lblue} scalar line integral}
        of f along C(t):

        \hspace{0.5cm}
        $\int_{C(t)}$ f ds = $\int_a^b$ f(C(t))$|| C'(t) ||$ dt
    \end{wtheorem}

    \begin{proof}
        Let partition P = \{$t_0,...,t_n$\} $\subset$ [a,b]
        with sample points C = \{$t_1^*,...,t_n^*$\},
        the Riemann sum of f along C(t) with partition P and sample points C:

        \hspace{0.5cm}
        $\sum_{i=1}^n$ $f(C(t_i^*)) \Delta s_i$
        \hspace{1cm}
        where $\Delta s_i$ = $\int_{t_{i-1}}^{t_i}$ $||C'(t)||$ dt

        Since C(t) is $C^1$ so $\Delta s_i$ is differentiable, then
        by the Mean Value Theorem, there is a $t_i^{**}$ $\in$ $[t_{i-1},t_i]$:

        \hspace{0.5cm}
        $\Delta s_i$ = $\int_{t_{i-1}}^{t_i}$ $||C'(t)||$ dt
        = $(t_i - t_{i-1})||C'(t_i^{**})||$
        = $||C'(t_i^{**})|| \Delta t_i$

        Thus:

        $\underset{||P|| \rightarrow 0}{\lim}$
            $\sum_{i=1}^n$ $f(C(t_i^*)) \Delta s_i$
        = $\underset{||P|| \rightarrow 0}{\lim}$
            $\sum_{i=1}^n$ $f(C(t_i^*)) ||C'(t_i^{**})|| \Delta t_i$
        = $\int_a^b$ f(C(t))$|| C'(t) ||$ dt
    \end{proof}

    \newpage



    \begin{example}
        Let f(x,y) = xy.
        
        Find the scalar line integral of f along C(t)
        = $(t,t^2)$ for t $\in$ [0,1]
    \end{example}

    \begin{tbox}
        $\int_{C(t)}$ f ds
        = $\int_0^1$ $t^3 \sqrt{1+4t^2}$ dt
        = $[\frac{1}{32}(\frac{2}{5}(1+4t^2)^{\frac{5}{2}}
                -\frac{2}{3}(1+4t^2)^{\frac{3}{2}})]_0^1$
        = $\frac{5^{5/2}+1}{120}$
    \end{tbox}

    \vspace{0.5cm}



    \begin{definition}{Vector Line Integral}{14cm}
        Let $C^1$ path C(t): [a,b] $\rightarrow$ $\mathbb{R}^n$
        and F: C([a,b]) $\subset$ X $\subset$ $\mathbb{R}^n$ $\rightarrow$
        $\mathbb{R}^n$ be continuous.

        Then the {\color{lblue} vector line integral}
        of F along C(t):

        \hspace{0.5cm}
        $\int_{C(t)}$ F $\cdot$ ds = $\int_a^b$ F(C(t)) $\cdot$ $C'(t)$ dt

        \vspace{0.3cm}



        {\color{red} Physical Understanding \#1:
        Vector Line Integrals = Total Work}

        \hspace{0.5cm}
        \begin{minipage}{13.5cm}
            In physics, Work = Force * Change in distance in a direction d

            \hspace{0.5cm}
            W = $||F||$ $\cos(\theta)$ $||\Delta x||$ 
            = $||F||$ $||\Delta x||$ $\cos(\theta)$
            = F $\cdot$ $\Delta x$

            Since $x'(t_0)$
            = $\lim_{\Delta t \rightarrow 0}$
                $\frac{x(\Delta t + t_0) - x(t_0)}{\Delta t}$
            = $\lim_{\Delta t \rightarrow 0}$
                $\frac{\Delta x}{\Delta t}$, then the work at $t_0$:

            \hspace{0.5cm}
            W $\approx$ $F(x(t_0))$ $\cdot$ x'($t_0$) $\Delta t$

            So the total work for all points $t_0$ in a path:

            \hspace{0.5cm}
            $\sum_{t_0 \in \text{path}}$ $F(x(t_0))$ $\cdot$ x'($t_0$) $\Delta t$
            $\approx$ $\int_a^b$ F(C(t)) $\cdot$ $C'(t)$ dt
        \end{minipage}

        \vspace{0.3cm}



        {\color{red} Physical Understanding \#2:
        Vector Line Integrals = Circulation of Fluid}

        \hspace{0.5cm}
        \begin{minipage}{13.5cm}
            Let F measure the flow of a fluid

            \hspace{0.5cm}
            $\int_a^b$ F(C(t)) $\cdot$ $C'(t)$ dt
            = $\int_a^b$ F(C(t)) $\cdot$ $\frac{C'(t)}{||C'(t)||}$ $||C'(t)||$ dt

            \begin{itemize}[leftmargin=0.5cm, itemsep=0.01cm]
                \item F(C(t)) $\cdot$ $\frac{C'(t)}{||C'(t)||}$
                    = $||F(C(t))||$ $\frac{||C'(t)||}{||C'(t)||}$ $\cos(\theta)$
                    = $||F(C(t))||$ $\cos(\theta)$
                    is the component of F that moves in direction of C'(t)
                    which is tangent to C(t)

                \item $||C'(t)||$ dt is the change in arclength
            \end{itemize}

            Thus, $\int_a^b$ F(C(t)) $\cdot$
                            $\frac{C'(t)}{||C'(t)||}$ $||C'(t)||$ dt
            is the total change in fluid along C(t).

            Thus, the vector line integral is also called the
            {\color{lblue} circulation of F along C(t)}.
        \end{minipage}

        \vspace{0.3cm}



        If F(x,y,z) = (M(x,y,z),N(x,y,z),P(x,y,z)) and C(t) = (x(t),y(t),z(t)):

        \hspace{0.5cm}
        $\int_{C(t)}$ F $\cdot$ ds
        = $\int_a^b$ M(x,y,z) dx + N(x,y,z) dy + P(x,y,z) dz
    \end{definition}

    \vspace{0.5cm}



    \begin{example}
        Let F(x,y) = (x+y,x-y).
        Find the vector line integral of F along $x^2+y^2$ = 1.
    \end{example}

    \begin{tbox}
        C(t) = (cos(t),sin(t)) for t $\in$ $[0,2\pi]$.

        $\int_{C(t)}$ F $\cdot$ ds
        = $\int_0^{2\pi}$
            $(\cos(t)+\sin(t),\cos(t)-\sin(t)) \cdot (-\sin(t),\cos(t))$ dt

        \hspace{2cm}
        = $\int_0^{2\pi}$ $-2\sin(t)\cos(t) + \cos^2(t) - \sin^2(t)$ dt

        \hspace{2cm}
        = $\int_0^{2\pi}$ $-\sin(2t) + \cos(2t)$ dt
        = $[\frac{1}{2}\cos(2t) + \frac{1}{2}\sin(2t)]_0^{2\pi}$
        = 0
    \end{tbox}

    \vspace{0.5cm}



    \begin{definition}{Reparametrization for Paths}{14cm}
        Let $C_1$: [a,b] $\rightarrow$ $\mathbb{R}^n$ be a piecewise $C^1$ path.

        Then piecewise $C^1$ path $C_2$: [c,d] $\rightarrow$ $\mathbb{R}^n$
        is a {\color{lblue} reparametrization} of $C_1$ if there is a
        1-1 and onto v: [c,d] $\rightarrow$ [a,b] such that:

        \hspace{0.5cm}
        $C_2(t)$ = $C_1(v(t))$

        Thus, since continuous v(t) is 1-1 and onto, then either:

        \hspace{0.5cm}
        {\color{lblue} Orientation preserving}:
        \hspace{0.5cm}
        v(c) = a
        and v(d) = b
        \hspace{0.5cm}
        so v'(t) $\geq$ 0
        
        \hspace{0.5cm}
        {\color{lblue} Orientation reversing}:
        \hspace{0.7cm}
        v(c) = b
        and v(d) = a
        \hspace{0.5cm}
        so v'(t) $\leq$ 0
    \end{definition}

    \newpage



    \begin{wtheorem}{Scalar Line Integrals are unaffected
    by Reparametrization}{14cm}
        Let $C^1$ path $C_1(t)$: [a,b] $\rightarrow$ $\mathbb{R}^n$
        and f: C([a,b]) $\subset$ X $\subset$ $\mathbb{R}^n$ $\rightarrow$
        $\mathbb{R}$ be continuous.

        Let $C^1$ path $C_2(t)$: [c,d] $\rightarrow$ $\mathbb{R}^n$
        be a reparametrization of $C_1(t)$. Then:

        \hspace{0.5cm}
        $\int_{C_2(t)}$ f ds = $\int_{C_1(t)}$ f ds
    \end{wtheorem}

    \begin{proof}
        $\int_{C_2(t)}$ f ds
        = $\int_c^d$ f($C_2(t)$) $||C_2'(t)||$ dv
        = $\int_c^d$ f($C_1(v(t))$) $||C_1'(v(t)) v'(t)||$ dt

        \hspace{1.7cm}
        = $\int_c^d$ f($C_1(v(t))$) $||C_1'(v(t))||$ $|v'(t)|$ dt

        If v(t) is orientation preserving, then:

        \hspace{0.5cm}
        $\int_{C_2(t)}$ f ds
        = $\int_c^d$ f($C_1(v(t))$)$||C_1'(v(t))|| v'(t)$ dt
        = $\int_a^b$ f($C_1(v)$)$||C_1'(v)||$ dv

        \hspace{2.35cm}
        = $\int_{C_1(t)}$ f ds

        If v(t) is orientation reversing, then:

        \hspace{0.5cm}
        $\int_{C_2(t)}$ f ds
        = $\int_c^d$ $f(C_1(v(t)))||C_1'(v(t))|| \ [-v'(t)]$ dt
        = $- \int_b^a$ $f(C_1(v))||C_1'(v)||$ dv

        \hspace{2.35cm}
        = $\int_a^b$ $f(C_1(v))||C_1'(v)||$ dv
        = $\int_{C_1(t)}$ f ds
    \end{proof}

    \vspace{0.5cm}



    \begin{wtheorem}{Vector Line Integrals are affected by Reparametrization}{14cm}
        Let $C^1$ path $C_1(t)$: [a,b] $\rightarrow$ $\mathbb{R}^n$
        and F: $C_1$([a,b]) $\subset$ X $\subset$ $\mathbb{R}^n$ $\rightarrow$
        $\mathbb{R}^n$ be continuous.

        Let $C^1$ path $C_2(t)$: [c,d] $\rightarrow$ $\mathbb{R}^n$
        be a reparametrization of $C_1(t)$. Then:

        \begin{itemize}[itemsep=0.1cm]
            \item If $C_2(t)$ is orientation preserving:
            
                \hspace{0.5cm}
                $\int_{C_2(t)}$ F $\cdot$ ds = $\int_{C_1(t)}$ F $\cdot$ ds
            
            \item If $C_2(t)$ is orientation reversing:
            
                \hspace{0.5cm}
                $\int_{C_2(t)}$ F $\cdot$ ds = $-\int_{C_1(t)}$ F $\cdot$ ds
        \end{itemize}
    \end{wtheorem}

    \begin{proof}
        $\int_{C_2(t)}$ F $\cdot$ ds
        = $\int_c^d$ F($C_2(t)$) $\cdot$ $C_2'(t)$ dt
        = $\int_c^d$ F($C_1(v(t))$) $\cdot$ $C_1'(v(t)) v'(t)$ dt

        If v(t) is orientation preserving, then:

        \hspace{0.5cm}
        $\int_{C_2(t)}$ F $\cdot$ ds
        = $\int_a^b$ F($C_1(v)$) $\cdot$ $C_1'(v)$ dv
        = $\int_{C_1(t)}$ F $\cdot$ ds

        If v(t) is orientation reversing, then:

        \hspace{0.5cm}
        $\int_{C_2(t)}$ F $\cdot$ ds
        = $\int_b^a$ F($C_1(v)$) $\cdot$ $C_1'(v)$ dv
        = $-\int_{C_1(t)}$ F $\cdot$ ds
    \end{proof}

    \newpage



\subsection{ Green's Theorem }

    \begin{definition}{Simple, Closed Curves}{14cm}
        $C^1$ curve C(t): [a,b] $\rightarrow$ $\mathbb{R}^n$
        is {\color{lblue} simple} if C(t) have no self-intersections
        (i.e. 1-1) except possibly at endpoints.

        \vspace{0.3cm}

        C(t) is {\color{lblue} closed} if endpoints are the same
        (i.e. C(a) = C(b)).
    \end{definition}

    \vspace{0.5cm}



    \begin{wtheorem}{Green's Theorem}{14cm}
        Let D be a closed, bounded region in $\mathbb{R}^2$
        where the boundary of D, C consist of finitely many simple, closed,
        piecewise $C^1$ curves $C_i$.
        For each $C_i$, let parametrization $C_i(t)$ be such that
        as t increases, D is at the left of $C_i(t)$.

        Then for $C^1$ vector field F(x,y) = (M(x,y),N(x,y)):

        \hspace{0.5cm}
        $\oint_C$ F $\cdot$ ds
        = $\int \int_D$ $-\frac{\partial M}{\partial y}
                        + \frac{\partial N}{\partial x}$ dA
        = $\int \int_D$ $(\nabla \times F) \cdot e_3$ dA
    \end{wtheorem}

    \begin{proof}
        Note $\oint_{C(t)}$ F $\cdot$ ds = $\oint$ M(x,y) dx + N(x,y) dy.

        First, split C = $C_1$ + $C_2$ such that $C_1$ is the curve below
        and $C_2$ is the curve above.

        Parametrize $C_1(x)$: [a,b] $\rightarrow$ $\mathbb{R}^2$
        and $C_2(x)$: [a,b] $\rightarrow$ $\mathbb{R}^2$.

        \hspace{0.5cm}
        $\oint_{C}$ M(x,y) dx
        = $\int_{C_1}$ M(x,y) dx - $\int_{C_2}$ M(x,y) dx
        
        \hspace{3.1cm}
        = $\int_a^b$ M(x,$C_1(x)$) dx - $\int_a^b$ M(x,$C_2(x)$) dx

        \hspace{3.1cm}
        = $\int_a^b$ M(x,$C_1(x)$) - M(x,$C_2(x)$) dx

        Since
        $\int \int_D$ $-\frac{\partial M}{\partial y}$ dA
        = $\int_a^b \int_{C_1(x)}^{C_2(x)}$ $-\frac{\partial M}{\partial y}$ dydx
        = $\int_a^b$ $-M(x,C_2(x)) + M(x,C_1(x))$ dx

        then $\oint_{C(t)}$ M(x,y) dx
        = $\int \int_D$ $-\frac{\partial M}{\partial y}$ dA.

        \vspace{0.3cm}

        Next, split C = $C_3$ + $C_4$ such that $C_3$ is the curve to the left
        and $C_4$ is the curve to the right.
        Parametrize $C_3(y)$: [c,d] $\rightarrow$ $\mathbb{R}^2$
        and $C_4(y)$: [c,d] $\rightarrow$ $\mathbb{R}^2$.

        \hspace{0.5cm}
        $\oint_{C}$ N(x,y) dy
        = $-\int_{C_3}$ N(x,y) dy + $\int_{C_4}$ N(x,y) dy
        
        \hspace{3cm}
        = $-\int_c^d$ N($C_3(y)$,y) dy + $\int_c^d$ N($C_4(y)$,y) dy

        \hspace{3cm}
        = $\int_c^d$ -N($C_3(y)$,y) + N($C_4(y)$,y) dy

        Since
        $\int \int_D$ $\frac{\partial N}{\partial x}$ dA
        = $\int_c^d \int_{C_3(y)}^{C_4(y)}$ $\frac{\partial N}{\partial x}$ dxdy
        = $\int_c^d$ $N(C_4(y),y) - N(C_3(y),y)$ dy

        then $\oint_{C(t)}$ M(x,y) dx
        = $\int \int_D$ $-\frac{\partial M}{\partial y}$ dA.

        \vspace{0.3cm}

        Thus, $\oint_C$ F $\cdot$ ds
        = $\oint$ M(x,y) dx + N(x,y) dy
        = $\int \int_D$ $-\frac{\partial M}{\partial y}
                        + \frac{\partial N}{\partial x}$ dA
        and since
        
        $\nabla \times F$
        = $(0,0,\frac{\partial N}{\partial x}-\frac{\partial M}{\partial y})$,
        then
        $\int \int_D$ $-\frac{\partial M}{\partial y}
                        + \frac{\partial N}{\partial x}$ dA
        = $\int \int_D$ $(\nabla \times F) \cdot e_3$ dA
    \end{proof}

    \vspace{0.5cm}



    \begin{example}
        Find the area of an ellipse $\frac{x^2}{a^2} + \frac{y^2}{b^2}$ = 1.
    \end{example}

    \begin{tbox}
        The parametrization of an ellipse is C(t) = (a cos(t), b sin(t))
        for t $\in$ $[0,2\pi]$.

        Let F(x,y) = $\frac{-y}{2},\frac{x}{2}$. Thus:

        \hspace{0.5cm}
        Area
        = $\int \int_{\frac{x^2}{a^2} + \frac{y^2}{b^2} = 1}$ 1 dA
        = $\int \int_{\frac{x^2}{a^2} + \frac{y^2}{b^2} = 1}$
            $-\frac{\partial}{y}\frac{-y}{2} + \frac{\partial}{x}\frac{x}{2}$ dA
        = $\int_{C}$ F $\cdot$ ds

        \hspace{0.5cm}
        $\int_{C}$ F $\cdot$ ds
        = $\int_0^{2\pi}$
            ($-\frac{b}{2}$ sin(t), $\frac{a}{2}$ cos(t))
            $\cdot$ (-a sin(t), b cos(t)) dt

        \hspace{2.3cm}
        = $\int_0^{2\pi}$ $\frac{ab}{2}\sin^2(t) + \frac{ab}{2} \cos^2(t)$ dt
        = $\int_0^{2\pi}$ $\frac{ab}{2}$ dt
        = $ab \pi$
    \end{tbox}

    \newpage



    \begin{wtheorem}{Gauss's Theorem: Divergence Theorem in $\mathbb{R}^2$}{14cm}
        If Green's Thereom applies for region D where C is the boundary of D
        and n is a outward unit normal vector to D, then for $C^1$ vector field
        
        F(x,y) = (M(x,y),N(x,y)):

        \hspace{0.5cm}
        $\oint_C$ F $\cdot$ n ds
        = $\int \int_D$ $\nabla \cdot F$ dA
    \end{wtheorem}

    \begin{proof}
        Let C be parametrized by C(t) = $(x(t),y(t))$ for t $\in$ [a,b].
        Then the outward unit normal vector at t is
        n = $\frac{1}{\sqrt{(y'(t))^2+(-x'(t))^2}} (y'(t),-x'(t))$
        = $\frac{1}{||C'(t)||} (y'(t),-x'(t))$.

        \hspace{0.5cm}
        $\oint_C$ F $\cdot$ n ds
        = $\int_a^b$ (M(x(t),y(t)),N(x(t),y(t)))
                        $\cdot$ $\frac{1}{||C'(t)||} (y'(t),-x'(t))$
                        $||C'(t)||$ dt
        
        \hspace{2.6cm}
        = $\int_a^b$ M(x(t),y(t))y'(t) dt - 
            $\int_a^b$ N(x(t),y(t))x'(t) dt

        \hspace{2.6cm}
        = $\int_C$ M(x,y) dy - 
            $\int_C$ N(x,y) dx
        = $\int_C$ (-N,M) $\cdot$ ds

        By {\color{red} theorem 5.4.2}:

        \hspace{0.5cm}
        $\int_C$ (-N,M) $\cdot$ ds
        = $\int \int_D$ $\frac{\partial N}{\partial y}
                        + \frac{\partial M}{\partial x}$ dA
        = = $\int \int_D$ $\nabla \cdot F$ dA
    \end{proof}

    \vspace{0.5cm}



    \begin{example}
        Let F(x,y) = (-y,x) and region D: $x^2 + y^2$ $\leq$ 1.
        Find $\int \int_D$ div(F) dA.
    \end{example}

    \begin{tbox}
        Since $\int \int_D$ div(F) dA
        = $\int \int_D$ $\nabla \cdot F$ dA
        where the boundary of D, C(t) = (cos(t),sin(t)) for t $\in$ $[0,2\pi]$
        so C'(t) = (-sin(t),cos(t)) and n = (cos(t),sin(t)). Thus:

        \hspace{0.5cm}
        $\int \int_D$ div(F) dA
        = $\oint_C$ F $\cdot$ n ds

        \hspace{0.5cm}
        = $\int_0^{2\pi}$ $(-\sin(t),\cos(t)) \cdot (\cos(t),\sin(t))$
                            $\sqrt{(-\sin(t))^2+(\cos(t))^2}$ dt

        \hspace{0.5cm}
        = $\int_0^{2\pi}$ 0 dt
        = 0
    \end{tbox}

    \vspace{0.5cm}





\subsection{ Conservative Vector Fields }

    \begin{definition}{Path Independence}{14cm}
        A continuous vector field F has {\color{lblue} path independent}
        line integrals if for any two simple, piecewise $C^1$, oriented curves
        $C_1,C_2$ on [a,b] where $C_1(a)$ = $C_2(a)$
        and $C_1(b)$ = $C_2(b)$:

        \hspace{0.5cm}
        $\int_{C_1}$ F $\cdot$ ds = $\int_{C_2}$ F $\cdot$ ds
    \end{definition}

    \vspace{0.5cm}



    \begin{wtheorem}{Path Independence $\rightleftharpoons$
    $\int_{C_1}$ F $\cdot$ ds = 0}{14cm}
        Continuous vector field F has path independent line integrals
        if and only if
        
        $\oint_{C}$ F $\cdot$ ds = 0
        for all simple, piecewise $C^1$, closed curves.
    \end{wtheorem}

    \begin{proof}
        Suppose F has path independence.

        For simple, piecewise $C^1$, closed curve C,
        split C into $C_1(t)$,$C_2(t)$ for t $\in$ [a,b] such that
        $C_1(a)$ = $C_2(b)$ and $C_1(b)$ = $C_2(a)$.
        Since F has path independence:

        \hspace{0.5cm}
        $\oint_{C(t)}$ F $\cdot$ ds
        = $\int_{C_1(t)}$ F $\cdot$ ds - $\int_{C_2(t)}$ F $\cdot$ ds
        = 0

        \rule[0.1cm]{15.1cm}{0.01cm}

        Suppose $\oint_{C}$ F $\cdot$ ds = 0 for all simple, piecewise $C^1$,
        closed curves C.

        Then for simple, piecewise $C^1$ curves $C_1(t)$,$C_2(t)$
        for t $\in$ [a,b] such that $C_1(a)$ = $C_2(a)$ and $C_1(b)$ = $C_2(b)$,
        a simple, piecewise $C^1$, closed curve C = $C_1$ - $C_2$. Thus:

        \hspace{0.5cm}
        0 = $\oint_{C}$ F $\cdot$ ds
        = $\int_{C_1}$ F $\cdot$ ds - $\int_{C_2}$ F $\cdot$ ds
        \hspace{0.5cm}
        $\Rightarrow$
        \hspace{0.5cm}
        $\int_{C_1}$ F $\cdot$ ds = $\int_{C_2}$ F $\cdot$ ds
    \end{proof}

    \newpage



    \begin{definition}{Potential of F}{14cm}
        Let continuous vector field F: $\mathbb{R}^n$ $\rightarrow$ $\mathbb{R}^n$
        have a $C^1$ function f: $\mathbb{R}^n$ $\rightarrow$ $\mathbb{R}$
        such that $\nabla f$ = F.
        Then f is the {\color{lblue} potential} of F
        and F is a {\color{lblue} conservative vector field}.
    \end{definition}

    \vspace{0.5cm}



    \begin{wtheorem}{Fundamental Theorem of Line Integrals}{14cm}
        Let continuous vector field F: $\mathbb{R}^n$ $\rightarrow$ $\mathbb{R}^n$.
        Then F = $\nabla f$ if and only if F has path independent
        line integrals.

        If F = $\nabla f$ for some $C^1$ function f, then
        for any piecewise $C^1$, oriented curve C with initial point A and
        terminal point B:

        \hspace{0.5cm}
        $\int_C$ F $\cdot$ ds
        = f(B) - f(A)
    \end{wtheorem}

    \begin{proof}
        Suppose F = $\nabla f$ for some $C^1$ function f.

        Then Df(x) exist so
        $\frac{d}{dt} f(C(t))$ = Df(C(t)) C'(t) = $\nabla f(C(t))$ $\cdot$ C'(t)
        for any piecewise $C^1$, oriented curve C.
        Let C have initial point (a,A) and terminal point (b,B). Then:

        \hspace{0.5cm}
        $\int_C$ F $\cdot$ ds
        = $\int_a^b$ $\nabla f(C(t))$ $\cdot$ C'(t) dt
        = $\int_a^b$ $\frac{d}{dt} f(C(t))$ dt
        = [f(C(t))]$_a^b$
        = f(B) - f(A)

        Since $\int_C$ F $\cdot$ ds = f(B) - f(A) regardless of
        any piecewise $C^1$, oriented curve C(t), then F has path independence.

        \rule[0.1cm]{15.1cm}{0.01cm}

        Suppose F has path independence.

        Take n = 2. The proof is analogous for any n.

        For F = ($F_1,F_2$), let
        f(x,y)
        = $\int_C$ F $\cdot$ ds
        = $\int_{(0,0)}^{(x,y)}$ $F_1$ dx + $F_2$ dy
        where C is any piecewise $C^1$ curve from (0,0) to (x,y).
        Since F is path independent, then:

        \hspace{0.5cm}
        $\int_{(0,0)}^{(x,y)}$ $F_1$ dx + $F_2$ dy
        = [$\int_{(0,0)}^{(0,y)}$ $F_1$ dx + $F_2$ dy]
            + [$\int_{(0,y)}^{(x,y)}$ $F_1$ dx + $F_2$ dy]

        \hspace{0.5cm}
        = $\int_0^y$ $F(0,t) \cdot (0,1)$ dt + $\int_0^x$ $F(t,y) \cdot (1,0)$ dt
        = $\int_0^y$ $F_2(0,t)$ dt + $\int_0^x$ $F_1(t,y)$ dt

        By the Fundamental Theorem of Calculus,
        $\frac{\partial f}{\partial x}$ = 0 + $F_1(x,y)$ = $F_1(x,y)$.
        Also:

        \hspace{0.5cm}
        $\int_{(0,0)}^{(x,y)}$ $F_1$ dx + $F_2$ dy
        = [$\int_{(0,0)}^{(x,0)}$ $F_1$ dx + $F_2$ dy]
            + [$\int_{(x,0)}^{(x,y)}$ $F_1$ dx + $F_2$ dy]

        \hspace{0.5cm}
        = $\int_0^x$ $F(t,0) \cdot (1,0)$ dt + $\int_0^y$ $F(x,t) \cdot (0,1)$ dt
        = $\int_0^x$ $F_1(t,0)$ dt + $\int_0^y$ $F_2(x,t)$ dt

        By the Fundamental Theorem of Calculus,
        $\frac{\partial f}{\partial y}$ = 0 + $F_1(x,y)$ = $F_2(x,y)$.

        Thus, $\nabla f$ = ($F_1,F_2$) = F.
    \end{proof}

    \vspace{0.5cm}



    \begin{example}
        Recall the example at 5.3.2.
        Let F(x,y) = (x+y,x-y).
        
        Find the vector line integral of F along
        $x^2+y^2$ = 1 from (1,0) to (0,1).

        Also, justify why $\int_C$ F $\cdot$ ds = 0 over the unit circle.
    \end{example}

    \begin{tbox}
        Since F = $\nabla f$ where f = $\frac{1}{2}x^2 + xy - \frac{1}{2}y^2$,
        then:

        \hspace{0.5cm}
        $\int_C$ F $\cdot$ ds
        = f(0,1) - f(1,0) = $-\frac{1}{2}$ - $\frac{1}{2}$ = -1

        Additionally, the vector line integral of F along the entire $x^2+y^2$ = 1
        which is closed so initial point A is equal to terminal point B:

        \hspace{0.5cm}
        $\int_C$ F $\cdot$ ds = f(B) - f(A) = f(A) - f(A) = 0

        which is exactly the value calculated in the previous example.
    \end{tbox}




