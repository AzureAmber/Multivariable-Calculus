\newpage

\section[Day 5: Line Integrals]{ Line Integrals }

\subsection{ Parametrized Curves }

    \begin{definition}{Path}{14cm}
        A {\color{lblue} path} in $\mathbb{R}^n$ is a continuous
        C(t): [a,b] $\rightarrow$ $\mathbb{R}^n$.

        \vspace{0.3cm}
        If C(t) is twice-differentiable, then the
        {\color{lblue} velocity} of C(t) at $t_0$ $\in$ [a,b]:

        \hspace{0.5cm}
        v($t_0$) = x'($t_0$)
        
        Also, the aceleration of C(t) at $t_0$ $\in$ [a,b]:

        \hspace{0.5cm}
        a($t_0$) = x''($t_0$)
    \end{definition}

    \vspace{0.5cm}



    \begin{definition}{Arclength}{14cm}
        The length of a $C^1$ path C(t): [a,b] $\rightarrow$ $\mathbb{R}^n$:

        \hspace{0.5cm}
        L = $\int_a^b$ $||C'(t)||$ dt
    \end{definition}

    \begin{proof}
        Choose \{$x_1,...,x_n$\} $\in$ [a,b] such that each $x_i$ $<$ $x_{i+1}$.
        Then the length of C(t):

        \hspace{0.5cm}
        L = $\lim_{n \rightarrow \infty}$ $\sum_{i=1}^n$ $||C(x_i) - C(x_{i-1})||$

        Let each C($x_i$) = $(C_1(x_i),...,C_n(x_i))$. Thus:
        
        \hspace{0.5cm}
        $||C(x_i) - C(x_{i-1})||$
        = $\sqrt{(C_1(x_i)-C_1(x_{i-1}))^2 + ... + (C_n(x_i)-C_n(x_{i-1}))^2}$

        Since C(t) is $C^1$, by the Mean Value Theorem,
        there is a $t_{i_k}$ $\in$ $[x_{i+1},x_i]$ such that:

        \hspace{0.5cm}
        $C_k(x_i) - C_k(x_{i-1})$
        = $(x_i - x_{i-1})C_1'(t_{i_k})$

        Thus:

        \hspace{0.5cm}
        $||C(x_i) - C(x_{i-1})||$
        = $\sqrt{(x_i-x_{i-1})^2[C_1'(t_{i_1})]^2
                + ... + (x_i-x_{i-1})^2[C_1'(t_{i_n})]^2}$

        \hspace{0.5cm}
        L = $\lim_{n \rightarrow \infty}$ $\sum_{i=1}^n$
            $\sqrt{[C_1'(t_{i_1})]^2 + ... + [C_1'(t_{i_n})]^2}(x_i - x_{i-1})$
        
        \hspace{0.9cm}
        = $\int_a^b$ $\sqrt{[C_1'(t)]^2 + ... + [C_n'(t)]^2}$ dt
        = $\int_a^b$ $||C'(t)||$ dt 
    \end{proof}

    \vspace{0.5cm}





\subsection{ Vector Fields }

    \begin{definition}{Vector Field and Flow Lines}{14cm}
        A {\color{lblue} vector field} on $\mathbb{R}^n$ is
        F: X $\subset$ $\mathbb{R}^n$ $\rightarrow$ $\mathbb{R}^n$

        \vspace{0.5cm}

        A {\color{lblue} flow line} of vector field F
        is a differentiable path C(t): [a,b] $\rightarrow$ $\mathbb{R}^n$
        such that:

        \hspace{0.5cm}
        C'(t) = F(C(t))
    \end{definition}

    \vspace{0.5cm}



    \begin{definition}{Del Operator}{14cm}
        The {\color{lblue} Del Operator} on $\mathbb{R}^n$:
        
        \hspace{0.5cm}
        $\nabla$ = $\sum_{i=1}^n$ $\frac{\partial}{\partial x_i}e_i$
        = $(\frac{\partial}{\partial x_1},...,\frac{\partial}{\partial x_n})$
    \end{definition}

    \vspace{0.5cm}



    \begin{definition}{Divergence}{14cm}
        For differentiable vector field F: X $\subset$ $\mathbb{R}^n$
        $\rightarrow$ $\mathbb{R}^n$, let F = $(F_1,...,F_n)$.

        Then the {\color{lblue} divergence}, div:
        $\mathbb{R}^n$ $\rightarrow$ $\mathbb{R}$, of F:

        \hspace{0.5cm}
        div(F)
        = $\nabla$ $\cdot$ F
        = $\frac{\partial F_1}{\partial x_1}
                + ... + \frac{\partial F_n}{\partial x_n}$

        If div(F) = 0 everywhere, then F is incompressible.
    \end{definition}

    \newpage



    \begin{definition}{Curl}{14cm}
        For differentiable vector field F: X $\subset$ $\mathbb{R}^3$
        $\rightarrow$ $\mathbb{R}^3$, let F = $(F_1,F_2,F_3)$.

        Then the {\color{lblue} curl}, curl:
        $\mathbb{R}^n$ $\rightarrow$ $\mathbb{R}^n$, of F: 

        \hspace{0.5cm}
        curl(F)
        = $\nabla$ $\times$ F =
        $
        \begin{bmatrix}
            e_1 & e_2 & e_3 \\
            \frac{\partial}{\partial x} & \frac{\partial}{\partial y}
                & \frac{\partial}{\partial z} \\
            F_1 & F_2 & F_3
        \end{bmatrix}
        $
        = ($\frac{\partial F_3}{\partial y}-\frac{\partial F_2}{\partial z} \ , \
            \frac{\partial F_1}{\partial z}-\frac{\partial F_3}{\partial x} \ , \
            \frac{\partial F_2}{\partial x}-\frac{\partial F_1}{\partial y}$)

        If curl(F) = 0 everywhere, then F is irrotational.
    \end{definition}

    \vspace{0.5cm}



    \begin{wtheorem}{Vector fields from Gradients are irrotational}{14cm}
        If f: X $\subset$ $\mathbb{R}^3$ $\rightarrow$ $\mathbb{R}$
        is $C^2$, then curl($\nabla f$) = 0
    \end{wtheorem}

    \begin{proof}
        Since $\nabla f$
        = $(\frac{\partial f}{\partial x},
            \frac{\partial f}{\partial y},
            \frac{\partial f}{\partial z})$,
        then by {\color{red} theorem 2.3.9}:

        \hspace{0.5cm}
        curl($\nabla f$) =
        $
        \begin{bmatrix}
            e_1 & e_2 & e_3 \\
            \frac{\partial}{\partial x} & \frac{\partial}{\partial y}
                & \frac{\partial}{\partial z} \\
            \frac{\partial f}{\partial x} & \frac{\partial f}{\partial y}
                & \frac{\partial f}{\partial z}
        \end{bmatrix}
        $
        = ($\frac{\partial^2 f}{\partial y\partial z}
                - \frac{\partial^2 f}{\partial z\partial y} \ , \
            \frac{\partial^2 f}{\partial z\partial x}
                - \frac{\partial^2 f}{\partial x\partial z} \ , \
            \frac{\partial^2 f}{\partial x\partial y}
                - \frac{\partial^2 f}{\partial y\partial x}$)
        = 0
    \end{proof}

    \vspace{0.5cm}



    \begin{wtheorem}{The curl is incompressible}{14cm}
        If F: X $\subset$ $\mathbb{R}^3$ $\rightarrow$ $\mathbb{R}^3$
        is $C^2$, then div(curl(F)) = 0
    \end{wtheorem}

    \begin{proof}
        Since curl(F)
        = ($\frac{\partial F_3}{\partial y}-\frac{\partial F_2}{\partial z} \ , \
            \frac{\partial F_1}{\partial z}-\frac{\partial F_3}{\partial x} \ , \
            \frac{\partial F_2}{\partial x}-\frac{\partial F_1}{\partial y}$).
        then by {\color{red} theorem 2.3.9}:

        \hspace{0.5cm}
        div(curl(F))
        = $\frac{\partial}{\partial x}
            (\frac{\partial F_3}{\partial y}-\frac{\partial F_2}{\partial z})$
            + $\frac{\partial}{\partial y}
            (\frac{\partial F_1}{\partial z}-\frac{\partial F_3}{\partial x})$
            + $\frac{\partial}{\partial z}
            (\frac{\partial F_2}{\partial x}-\frac{\partial F_1}{\partial y})$
        = 0
    \end{proof}

    \vspace{0.5cm}





\subsection{ Scalar \& Vector Line Integrals }

    \begin{wtheorem}{Scalar Line Integral}{14cm}
        Let $C^1$ path C(t): [a,b] $\rightarrow$ $\mathbb{R}^n$
        and f: C([a,b]) $\subset$ X $\subset$ $\mathbb{R}^n$ $\rightarrow$
        $\mathbb{R}$ be continuous.

        Then the {\color{lblue} scalar line integral}
        of f along C(t):

        \hspace{0.5cm}
        $\int_{C(t)}$ f ds = $\int_a^b$ f(C(t))$|| C'(t) ||$ dt
    \end{wtheorem}

    \begin{proof}
        Let partition P = \{$t_0,...,t_n$\} $\subset$ [a,b]
        with sample points C = \{$t_1^*,...,t_n^*$\},
        the Riemann sum of f along C(t) with partition P and sample points C:

        \hspace{0.5cm}
        $\sum_{i=1}^n$ $f(C(t_i^*)) \Delta s_i$
        \hspace{1cm}
        where $\Delta s_i$ = $\int_{t_{i-1}}^{t_i}$ $||C'(t)||$ dt

        Since C(t) is $C^1$ so $\Delta s_i$ is differentiable, then
        by the Mean Value Theorem, there is a $t_i^{**}$ $\in$ $[t_{i-1},t_i]$:

        \hspace{0.5cm}
        $\Delta s_i$ = $\int_{t_{i-1}}^{t_i}$ $||C'(t)||$ dt
        = $(t_i - t_{i-1})||C'(t_i^{**})||$
        = $||C'(t_i^{**})|| \Delta t_i$

        Thus:

        $\underset{||P|| \rightarrow 0}{\lim}$
            $\sum_{i=1}^n$ $f(C(t_i^*)) \Delta s_i$
        = $\underset{||P|| \rightarrow 0}{\lim}$
            $\sum_{i=1}^n$ $f(C(t_i^*)) ||C'(t_i^{**})|| \Delta t_i$
        = $\int_a^b$ f(C(t))$|| C'(t) ||$ dt
    \end{proof}

    \newpage





    \begin{wtheorem}{Vector Line Integral}{14cm}
        Let $C^1$ path C(t): [a,b] $\rightarrow$ $\mathbb{R}^n$
        and F: C([a,b]) $\subset$ X $\subset$ $\mathbb{R}^n$ $\rightarrow$
        $\mathbb{R}^n$ be continuous.

        Then the {\color{lblue} vector line integral}
        of f along C(t):

        \hspace{0.5cm}
        $\int_{C(t)}$ F $\cdot$ ds = $\int_a^b$ F(C(t)) $\cdot$ $C'(t)$ dt

        If F: $\mathbb{R}^3$ $\rightarrow$ $\mathbb{R}^3$, let
        F(x,y,z) = (M(x,y,z),N(x,y,z),P(x,y,z)) and

        C(t) = (x(t),y(t),z(t)), then:

        \hspace{0.5cm}
        $\int_{C(t)}$ F $\cdot$ ds
        = $\int_a^b$ M(x,y,z) dx + N(x,y,z) dy + P(x,y,z) dz
    \end{wtheorem}

    \vspace{0.5cm}





\subsection{ Green's Theorem }

    \begin{wtheorem}{Green's Theorem}{14cm}
        Let D be a closed, bounded region in $\mathbb{R}^2$
        where the boundary of D, C consist of finitely many simple, closed,
        piecewise $C^1$ curves $C_i$.
        For each $C_i$, let parametrization $C_i(t)$ be such that
        as t increases, D is at the left of $C_i(t)$.

        Then for $C^1$ vector field F(x,y) = (M(x,y),N(x,y)):

        \hspace{0.5cm}
        $\oint_C$ F $\cdot$ ds
        = $\int \int_D$ $-\frac{\partial M}{\partial y}
                        + \frac{\partial N}{\partial x}$ dA
        = $\int \int_D$ $(\nabla \times F) \cdot e_3$ dA
    \end{wtheorem}

    \vspace{0.5cm}



    \begin{wtheorem}{Gauss's Theorem: Divergence Theorem}{14cm}
        If Green's Thereom applies for region D and n is a outward normal
        unit vector to D, then for $C^1$ vector field F(x,y) = (M(x,y),N(x,y)):

        \hspace{0.5cm}
        $\oint_C$ F $\cdot$ n ds
        = $\int \int_D$ $\nabla \cdot F$ dA
        
    \end{wtheorem}

    \vspace{0.5cm}





\subsection{ Conservative Vector Fields }

    \begin{definition}{Path Independence}{14cm}
        A continuous vector field F has {\color{lblue} path independent}
        line integrals if for any two simple, piecewise $C^1$, oriented curves
        $C_1,C_2$ on [a,b] where $C_1(a)$ = $C_2(a)$
        and $C_1(b)$ = $C_2(b)$:

        \hspace{0.5cm}
        $\int_{C_1}$ F $\cdot$ ds = $\int_{C_2}$ F $\cdot$ ds
    \end{definition}

    \vspace{0.5cm}



    \begin{wtheorem}{Path Independence $\rightleftharpoons$
    $\int_{C_1}$ F $\cdot$ ds = 0}{14cm}
        For continuous vector field F, F has path independent line integrals
        if and only if $\int_{C}$ F $\cdot$ ds = 0
        for all simple, piecewise $C^1$, closed curves.
    \end{wtheorem}

    \vspace{0.5cm}



    \begin{definition}{Potential of F}{14cm}
        Let continuous vector field F: $\mathbb{R}^n$ $\rightarrow$ $\mathbb{R}^n$
        have a $C^1$ function f: $\mathbb{R}^n$ $\rightarrow$ $\mathbb{R}$
        such that $\nabla f$ = F.
        Then f is the {\color{lblue} potential} of F
        and F is a {\color{lblue} conservative vector field}.
    \end{definition}

    \vspace{0.5cm}



    \begin{wtheorem}{Conservative Vector Fields $\rightleftharpoons$
    Path independence}{14cm}
        Let continuous vector field F: $\mathbb{R}^n$ $\rightarrow$ $\mathbb{R}^n$.
        Then F = $\nabla f$ if and only if F has path independent
        line integrals.

        If F = $\nabla f$ for some $C^1$ function f, then
        for any piecewise $C^1$, oriented curve C with initial point A and
        terminal point B:

        \hspace{0.5cm}
        $\int_C$ F $\cdot$ ds
        = f(B) - f(A)
    \end{wtheorem}




