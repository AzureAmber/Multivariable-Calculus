\newpage

\section[Day 2: Differentiation]{ Differentiation }

\subsection{ Limits \& Continuity }

    \begin{definition}{Limit}{14cm}
        For f: X $\subset$ $\mathbb{R}^n$ $\rightarrow$ $\mathbb{R}^m$,
        let a $\in$ X.

        If for every $\epsilon > 0$, there is a $\delta > 0$ such that
        for all x $\in$ X where $||x-a|| < \delta$:

        \hspace{0.5cm}
        $||f(x) - L|| < \epsilon$

        Then the {\color{lblue} limit} of f(x) as x approaches a is
        $\lim_{x \rightarrow a}$ f(x) = L.
    \end{definition}

    \vspace{0.5cm}



    \begin{example}
        Let f(x,y) = $2x^2 + xy$. Find f(x,y) as (x,y) $\rightarrow$ (-1,1).
    \end{example}

    \begin{tbox}
        L = f(-1,1) = 1.
        Let $\sqrt{(x+1)^2 + (y-1)^2}$ $<$ $\delta$
        so $|x+1| < \delta$ and $|y-1| < \delta$. Thus:

        \hspace{0.5cm}
        $|f(x,y) - L|$
        = $|2x^2 + xy - 1|$
        = $|2x^2-2 + xy+1|$

        \hspace{2.85cm}
        = $|2(x+1)(x-1) + (x+1)(y+1)-(x+1+y-1)|$
        
        \hspace{2.85cm}
        $\leq$ $2|x+1|*|x-1| + |x+1|*|y+1| + |x+1| + |y-1|$

        \hspace{2.85cm}
        $<$ $2\delta(\delta+2) + \delta(\delta + 2) + 2\delta$
        = $3\delta^2 + 8\delta$

        Since min($3\delta^2 + 8\delta$) = $\frac{-16}{3} < 0$, then
        for any $\epsilon > 0$, there is a $\delta$ where
        $3\delta^2 + 8\delta < \epsilon$.
        
        Thus, $|f(x) - L| < 3\delta^2 + 8\delta < \epsilon$.
    \end{tbox}

    \vspace{0.5cm}



    \begin{wtheorem}{Limits are Unique}{14cm}
        If $\lim_{x \rightarrow a}$ f(x) = $L_1$
        and $\lim_{x \rightarrow a}$ f(x) = $L_2$, then $L_1$ = $L_2$.
    \end{wtheorem}

    \begin{proof}
        Since $\lim_{x \rightarrow a}$ f(x) = $L_1$, there is a $\delta_1$
        where for $||x-a|| < \delta_1$, then
        $||f(x) - L_1||$ $<$ $\frac{\epsilon}{2}$.

        Since $\lim_{x \rightarrow a}$ f(x) = $L_2$, there is a $\delta_2$
        where for $||x-a|| < \delta_2$, then
        $||f(x) - L_2||$ $<$ $\frac{\epsilon}{2}$.

        Let $\delta$ = min($\delta_1,\delta_2$).
        Then for $||x-a|| < \delta$:

        \hspace{0.5cm}
        $||L_1 - L_2||$
        $\leq$ $||L_1 - f(x)|| + ||f(x) - L_2||$
        $<$ $\frac{\epsilon}{2} + \frac{\epsilon}{2}$
        = $\epsilon$
    \end{proof}

    \vspace{0.5cm}



    \begin{ltheorem}{Properties of the Limit}{1.5cm}
        \item For f,g: $\mathbb{R}^n$ $\rightarrow$ $\mathbb{R}^m$, if
            $\lim_{x \rightarrow a}$ f(x) = A and
            $\lim_{x \rightarrow a}$ g(x) = B, then:

            \hspace{0.5cm}
            $\lim_{x \rightarrow a}$ (f+g)(x) = A+B

            \begin{proof}[14cm]
                Since $\lim_{x \rightarrow a}$ f(x) = A, there is a $\delta_1$
                where for $||x-a|| < \delta_1$, then:

                \hspace{0.5cm}
                $||f(x) - A||$ $<$ $\frac{\epsilon}{2}$

                Since $\lim_{x \rightarrow a}$ g(x) = B, there is a $\delta_2$
                where for $||x-a|| < \delta_2$, then:

                \hspace{0.5cm}
                $||g(x) - B||$ $<$ $\frac{\epsilon}{2}$

                Let $\delta$ = min($\delta_1,\delta_2$).
                Then for $||x-a|| < \delta$:

                \hspace{0.5cm}
                $||(f+g)(x) - (A+B)||$
                = $||f(x)+g(x) - A - B||$

                \hspace{0.5cm}
                $\leq$ $||f(x) - A|| + ||g(x) - B||$
                $<$ $\frac{\epsilon}{2} + \frac{\epsilon}{2}$
                = $\epsilon$
            \end{proof}

        \item For f: $\mathbb{R}^n$ $\rightarrow$ $\mathbb{R}^m$, if
            $\lim_{x \rightarrow a}$ f(x) = A and scalar c $\in$ $\mathbb{R}$, then:
        
            \hspace{0.5cm}
            $\lim_{x \rightarrow a}$ cf(x) = cA

            \begin{proof}[14cm]
                Since $\lim_{x \rightarrow a}$ f(x) = A, there is a $\delta$
                where for $||x-a|| < \delta$, then:

                \hspace{0.5cm}
                $||f(x) - A||$ $<$ $\frac{\epsilon}{c}$

                Then,
                $||cf(x) - cA||$
                = $c||f(x) - A||$
                $<$ $c\frac{\epsilon}{c}$
                = $\epsilon$.
            \end{proof}

            \newpage

        \item For f,g: $\mathbb{R}^n$ $\rightarrow$ $\mathbb{R}$, if
            $\lim_{x \rightarrow a}$ f(x) = A and
            $\lim_{x \rightarrow a}$ g(x) = B, then:

            \hspace{0.5cm}
            $\lim_{x \rightarrow a}$ (fg)(x) = AB

            \begin{proof}[14cm]
                Note 4fg = $(f+g)^2 - (f-g)^2$.

                By part (a), there is a $\delta$ where for $||x-a|| < \delta$:

                \hspace{0.5cm}
                $|(f+g)(x) - (A+B)| < \epsilon$

                Then as x $\rightarrow$ a:

                \hspace{0.5cm}
                $| [(f+g)(x)]^2 - [A+B]^2 |$
                = $| [(f+g)(x) - (A+B)][(f+g)(x) + (A+B)] |$

                \hspace{0.5cm}
                = $| (f+g)(x) - (A+B) | * |(f+g)(x) + (A+B)|$
                = $\epsilon(2(A+B))$

                Thus, $\lim_{x \rightarrow a}$ $(f+g)^2(x)$ = $(A+B)^2$.

                The proof for $\lim_{x \rightarrow a}$ $(f-g)^2(x)$ = $(A-B)^2$
                is analogous.
                Thus:

                \hspace{0.5cm}
                $\lim_{x \rightarrow a}$ (fg)(x)
                = $\lim_{x \rightarrow a}$ $\frac{1}{4}$[$(f+g)^2(x)$-$(f-g)^2(x)$]
                
                \hspace{0.5cm}
                = $\frac{1}{4}[(A+B)^2 - (A-B)^2]$
                = $\frac{1}{4}4AB$
                = AB
            \end{proof}

        \item For f,g: $\mathbb{R}^n$ $\rightarrow$ $\mathbb{R}$, if
            $\lim_{x \rightarrow a}$ f(x) = A and
            $\lim_{x \rightarrow a}$ g(x) = B $\not =$ 0, then:

            \hspace{0.5cm}
            $\lim_{x \rightarrow a}$ ($\frac{\text{f}}{\text{g}}$)(x)
            = $\frac{\text{A}}{\text{B}}$

            \begin{proof}[14cm]
                Since $\lim_{x \rightarrow a}$ g(x) = B, then there is a
                $\delta$ where for $||x-a|| < \delta$:

                \hspace{0.5cm}
                $|g(x) - B| < \epsilon$

                Thus, as x $\rightarrow$ a:

                \hspace{0.5cm}
                $|\frac{1}{g(x)} - \frac{1}{B}|$
                = $|\frac{B-g(x)}{Bg(x)}|$
                = $|B-g(x)| * |\frac{1}{Bg(x)}|$
                $<$ $\epsilon \frac{1}{B^2}$

                Thus, $\lim_{x \rightarrow a}$ $\frac{1}{g(x)}$ = $\frac{1}{B}$.
                By part (c), then
                $\lim_{x \rightarrow a}$ ($\frac{\text{f}}{\text{g}}$)(x)
                = $\frac{\text{A}}{\text{B}}$.
            \end{proof}
    \end{ltheorem}

    \vspace{0.5cm}



    \begin{wtheorem}{Components of Limits}{14cm}
        For f: X $\subset$ $\mathbb{R}^n$ $\rightarrow$ $\mathbb{R}^m$,
        let f(x) = $(f_1(x),...,f_m(x))$. Then for i = \{1,...,m\}:

        \hspace{0.5cm}
        $\lim_{x \rightarrow a}$ f(x) = L = $(L_1,...,L_m)$ if and only if each
        $\lim_{x \rightarrow a}$ $f_i(x)$ = $L_i$ 
    \end{wtheorem}

    \begin{proof}
        If $\lim_{x \rightarrow a}$ f(x) = L = $(L_1,...,L_m)$,
        then there is a $\delta$ such that for $||x-a|| < \delta$:

        \hspace{0.5cm}
        $|| f(x) - L || < \epsilon$

        \hspace{0.5cm}
        $|| (f_1(x),...,f_m(x)) - (L_1,...,L_m) ||$
        = $\sqrt{(f_1(x) - L_1)^2 + ... + (f_m(x) - L_m)^2}$
        $<$ $\epsilon$

        Thus, each $|f_i(x) - L_i| < \epsilon$ for $||x-a|| < \delta$
        so $\lim_{x \rightarrow a}$ $f_i(x)$ = $L_i$.

        \rule[0.1cm]{15cm}{0.01cm}

        If each $\lim_{x \rightarrow a}$ $f_i(x)$ = $L_i$ , then there
        are $\delta_i$ such that for $||x-a|| < \delta_i$:

        \hspace{0.5cm}
        $|f_i(x) - L_i| < \frac{\epsilon}{\sqrt{m}}$

        Let $\delta$ = min($\delta_1,...,\delta_m$).
        Then for $||x-a|| < \delta$:

        \hspace{0.5cm}
        $|| f(x) - L ||$
        = $|| (f_1(x),...,f_m(x)) - (L_1,...,L_m) ||$

        \hspace{2.7cm}
        = $\sqrt{(f_1(x) - L_1)^2 + ... + (f_m(x) - L_m)^2}$
        $<$ $\sqrt{\sum_{i=1}^m (\frac{\epsilon}{\sqrt{m}})^2}$
        = $\sqrt{\epsilon^2}$
        = $\epsilon$
    \end{proof}

    \vspace{0.5cm}



    \begin{definition}{Continuity}{14cm}
        For f: X $\subset$ $\mathbb{R}^n$ $\rightarrow$ $\mathbb{R}^m$,
        let a $\in$ X.

        Then f is {\color{lblue} continuous} at a if
        $\lim_{x \rightarrow a}$ f(x) = f(a).

        If f is continuous at every x $\in$ X, then f is continuous on X.
    \end{definition}

    \newpage



    \begin{ltheorem}{Properties of Continuity}{1.5cm}
        \item If f,g: X $\subset$ $\mathbb{R}^n$ $\rightarrow$ $\mathbb{R}^m$
            are continuous at a $\in$ X, then f+g is continuous at a

            \begin{proof}[14cm]
                Since $\lim_{x \rightarrow a}$ f(x) = f(a)
                and $\lim_{x \rightarrow a}$ g(x) = g(a),
                by {\color{red} theorem 2.1.3(a)}, then A = f(a) and B = g(a).
                Thus, $\lim_{x \rightarrow a}$ (f+g)(x) = f(a)+f(b).
            \end{proof}

        \item If f: X $\subset$ $\mathbb{R}^n$ $\rightarrow$ $\mathbb{R}^m$
            is continuous at a $\in$ X and scalar c $\in$ $\mathbb{R}$,
            then cf is continuous at a

            \begin{proof}[14cm]
                Since $\lim_{x \rightarrow a}$ f(x) = f(a),
                by {\color{red} theorem 2.1.3(b)}, then A = f(a).

                Thus, $\lim_{x \rightarrow a}$ cf(x) = cf(a).
            \end{proof}

        \item If f,g: X $\subset$ $\mathbb{R}^n$ $\rightarrow$ $\mathbb{R}$
            are continuous at a $\in$ X, then fg is continuous at a

            \begin{proof}[14cm]
                Since $\lim_{x \rightarrow a}$ f(x) = f(a)
                and $\lim_{x \rightarrow a}$ g(x) = g(a),
                by {\color{red} theorem 2.1.3(c)}, then A = f(a) and B = g(a).
                Thus, $\lim_{x \rightarrow a}$ (fg)(x) = f(a)f(b).
            \end{proof}

        \item If f,g: X $\subset$ $\mathbb{R}^n$ $\rightarrow$ $\mathbb{R}$
            are continuous at a $\in$ X where g(x) $\not =$ 0, then
            $\frac{\text{f}}{\text{g}}$ is continuous at a

            \begin{proof}[14cm]
                Since $\lim_{x \rightarrow a}$ f(x) = f(a)
                and $\lim_{x \rightarrow a}$ g(x) = g(a),
                by {\color{red} theorem 2.1.3(d)}, then A = f(a)
                and B = g(a) $\not =$ 0.
                Thus, $\lim_{x \rightarrow a}$ ($\frac{\text{f}}{\text{g}}$)(x)
                = $\frac{\text{f(a)}}{\text{g(b)}}$.
            \end{proof}
    \end{ltheorem}

    \vspace{0.5cm}



    \begin{wtheorem}{Components of Continuity}{14cm}
        For f: X $\subset$ $\mathbb{R}^n$ $\rightarrow$ $\mathbb{R}^m$,
        let f(x) = $(f_1(x),...,f_m(x))$. Then for i = \{1,...,m\}:

        \hspace{0.5cm}
        f is continuous at a $\in$ X if and only if each $f_i$ is continuous
        at a
    \end{wtheorem}

    \begin{proof}
        If f is continuous at a, then
        $\lim_{x \rightarrow a}$ f(x) = f(a) = $(f_1(a),...,f_m(a))$.
        By {\color{red} theorem 2.1.4}, then L = $(f_1(a),...,f_m(a))$
        so each $L_i$ = $f_i(a)$. Thus, for each i = \{1,...,m\}:

        \hspace{0.5cm}
        $\lim_{x \rightarrow a}$ $f_i(x)$ = $L_i$ = $f_i(a)$

        \rule[0.1cm]{15cm}{0.01cm}

        If each $f_i$ is continuous at a, then for i = \{1,...,m\},
        $\lim_{x \rightarrow a}$ $f_i(x)$ = $f_i(a)$.
        By {\color{red} theorem 2.1.4}, then L = $(f_1(a),...,f_m(a))$. Thus:

        \hspace{0.5cm}
        $\lim_{x \rightarrow a}$ f(x) = L = $(f_1(a),...,f_m(a))$ = f(a)
    \end{proof}

    \vspace{0.5cm}



    \begin{wtheorem}{Composite of Continuous functions are Continuous}{14cm}
        If f: X $\subset$ $\mathbb{R}^n$ $\rightarrow$ $\mathbb{R}^m$
        and g: Y $\subset$ $\mathbb{R}^m$ $\rightarrow$ $\mathbb{R}^k$
        are continuous where f(X) $\subset$ Y, then g $\circ$ f = g(f):
        X $\subset$ $\mathbb{R}^n$ $\rightarrow$ $\mathbb{R}^k$ is continuous
    \end{wtheorem}

    \begin{proof}
        For any a $\in$ X and any $\delta > 0$,
        there is a $\eta > 0$ such that for $||x-a|| < \eta$:

        \hspace{0.5cm}
        $||f(x) - f(a)|| < \delta$

        Since f(X) $\subset$ Y, then for any x $\in$ X, then f(x) $\in$ Y.

        For any f(a) $\in$ Y and any $\epsilon > 0$,
        there is a $\delta > 0$ such that for $||y-f(a)|| < \delta$:

        \hspace{0.5cm}
        $||g(y) - g(f(a))|| < \epsilon$

        Thus, for $||x-a|| < \eta$, then
        $||g(f(x)) - g(f(a))|| < \epsilon$.
    \end{proof}

    \newpage





\subsection{ Differentiability }

    \begin{definition}{Partial Derivative}{14cm}
        For f: X $\subset$ $\mathbb{R}^n$ $\rightarrow$ $\mathbb{R}$,
        let x = $(x_1,...,x_n)$ $\in$ X.

        For i = \{1,...,n\}, the {\color{lblue} partial derivative} of f
        with respect to $x_i$:

        \hspace{0.5cm}
        $D_if$ = $\frac{\partial f}{\partial x_i}$ = $f_{x_i}(x)$
        = $\lim_{h \rightarrow 0}$
            $\frac{f(x_1,...,x_i+h,...,x_n) - f(x_1,...,x_n)}{h}$
    \end{definition}

    \vspace{0.5cm}



    \begin{wtheorem}{Tangent Plane}{14cm}
        For f: X $\subset$ $\mathbb{R}^2$ $\rightarrow$ $\mathbb{R}$,
        let z = f(x,y).
        
        The {\color{lblue} tangent plane} at (a,b,f(a,b))
        has an equation of the form:

        \hspace{0.5cm}
        z = f(a,b) + $f_x(a,b)(x-a)$ + $f_y(a,b)(y-b)$
    \end{wtheorem}

    \begin{proof}
        Since $f_x(a,b)$ which is the change in z for every change in x
        is a tangent vector to f in direction of x at (a,b), then
        (1,0,$f_x(a,b)$) is parallel to the tangent plane.
        Similarily, (0,1,$f_y(a,b)$) is parallel to the tangent plane.

        Thus, (1,0,$f_x(a,b)$) $\times$ (0,1,$f_y(a,b)$)
        = ($-f_x(a,b),-f_y(a,b),1$) is orthogonal to the tangent plane.
        Thus, for any (x,y,z) in the plane:

        \hspace{0.5cm}
        ($-f_x(a,b),-f_y(a,b),1$) $\cdot$ [(x,y,z) - (a,b,f(a,b))] = 0

        \hspace{0.5cm}
        $-f_x(a,b)(x-a) - f_y(a,b)(y-b) + z-f(a,b)$ = 0

        \hspace{0.5cm}
        z = f(a,b) + $f_x(a,b)(x-a)$ + $f_y(a,b)(y-b)$
    \end{proof}

    \vspace{0.5cm}



    \begin{definition}{Differentiability in $\mathbb{R}^2$
    $\rightarrow$ $\mathbb{R}$}{14cm}
        f: X $\subset$ $\mathbb{R}^2$ $\rightarrow$ $\mathbb{R}$
        is {\color{lblue} differentiable} at x $\in$ X if there is an
        A $\in$ $M_{1 \times 2}(\mathbb{R})$ such that for h $\in$ X:

        \hspace{0.5cm}
        $\lim_{h \rightarrow 0}$ $\frac{|f(x+h)-f(x)-Ah|}{||h||}$ = 0

        Then, the {\color{lblue} derivative} of f at x is Df(x) = A =
        $
        \begin{bmatrix}
            \frac{\partial f}{\partial x}(x,y)
            & \frac{\partial f}{\partial y}(x,y)
        \end{bmatrix}
        $.

        If f is differentiable at every x $\in$ X, then f is differentiable on X.
    \end{definition}

    \vspace{0.5cm}



    \begin{wtheorem}{Continuous partials imply Differentiability}{14cm}
        If f: X $\subset$ $\mathbb{R}^2$ $\rightarrow$ $\mathbb{R}$
        has continuous partial derivatives at (a,b),

        then f is differentiable at (a,b)
    \end{wtheorem}

    \begin{proof}
        Since $f_x(x,y),f_y(x,y)$ is continuous at (a,b), then for $\epsilon > 0$,
        there is a $\delta > 0$ where for $||(x,y)-(a,b)|| < \delta$:
        
        \hspace{0.5cm}
        $|f_x(x,y) - f_x(a,b)| < \epsilon$
        \hspace{1cm}
        $|f_y(x,y) - f_y(a,b)| < \epsilon$
        
        Then for h = $h_1e_1 + h_2e_2$:

        \hspace{0.5cm}
        $\lim_{h \rightarrow 0}$
            $\frac{|f(a+h_1,b+h_2) - f(a,b) - [f_x(a,b)h_1 + f_y(a,b)h_2]|}{||h||}$

        \hspace{0.5cm}
        = $\lim_{h \rightarrow 0}$
            $\frac{|f(a+h_1,b+h_2) - f(a+h_1,b) + f(a+h_1,b) - f(a,b)
                    - [f_x(a,b)h_1 + f_y(a,b)h_2]|}{||h||}$

        Since $f_x(x,y),f_y(x,y)$ exist, then by the Mean Value Theorem,
        there are $t_1$ $\in$ $(0,h_1)$ and $t_2$ $\in$ $(0,h_2)$ such that:

        \hspace{0.5cm}
        $f(a+h_1,b) - f(a,b)$
        = $h_1 * f_x(a+t_1,b)$

        \hspace{0.5cm}
        $f(a+h_1,b+h_2) - f(a+h_1,b)$
        = $h_2 * f_y(a+h,b+t_2)$

        Thus, for $||h-(a,b)||<\delta$:

        \hspace{0.5cm}
        $\lim_{h \rightarrow 0}$
            $\frac{|f(a+h_1,b+h_2) - f(a,b) - [f_x(a,b)h_1 + f_y(a,b)h_2]|}{||h||}$
        
        \hspace{0.5cm}
        = $\lim_{h \rightarrow 0}$
            $\frac{|h_2 * f_y(a+h,b+t_2) + h_1 * f_x(a+t_1,b)
                    - [f_x(a,b)h_1 + f_y(a,b)h_2]|}{||h||}$

        \hspace{0.5cm}
        = $\lim_{h \rightarrow 0}$
            $\frac{|h_2 * [f_y(a+h,b+t_2)-f_y(a,b)]
                    + h_1 * [f_x(a+t_1,b)-f_x(a,b)]|}{||h||}$
        $<$ $\lim_{h \rightarrow 0}$
            $\frac{||h||\epsilon + ||h||\epsilon}{||h||}$
        = $2\epsilon$
    \end{proof}

    \newpage



    \begin{wtheorem}{Differentiability implies Continuity}{14cm}
        If f: X $\subset$ $\mathbb{R}^2$ $\rightarrow$ $\mathbb{R}$
        is differentiable at (a,b), then f is continuous at (a,b)
    \end{wtheorem}

    \begin{proof}
        If f is differentiable at (a,b), then
        $\lim_{h \rightarrow 0}$ $\frac{|f((a,b)+h)-f(a,b)-Ah|}{||h||}$ = 0.

        Thus, as h $\rightarrow$ 0, then A = $\frac{f((a,b)+h) - f(a,b)}{||h||}$.
        So:
        
        \hspace{0.5cm}
        $f((a,b)+h) - f(a,b)$
        = $[f((a,b)+h) - f(a,b)] \frac{||h||}{||h||}$
        = $A||h||$
        $\rightarrow$ 0

        Thus, f is continuous at (a,b).
    \end{proof}

    \vspace{0.5cm}





\subsection{ Differentiability in Higher Dimensions }

    \begin{definition}{Differentiability in $\mathbb{R}^n$
    $\rightarrow$ $\mathbb{R}$}{14cm}
        Differentiability can be extended for $\mathbb{R}^n$.

        f: X $\subset$ $\mathbb{R}^n$ $\rightarrow$ $\mathbb{R}$
        is differentiable at x $\in$ X if there is an
        A $\in$ $M_{1 \times n}(\mathbb{R})$ such that for h $\in$ X:

        \hspace{0.5cm}
        $\lim_{h \rightarrow 0}$ $\frac{|f(x+h)-f(x)-Ah|}{||h||}$ = 0

        Then, the derivative of f at x is Df(x) = A =
        $
        \begin{bmatrix}
            \frac{\partial f}{\partial x_1}(x)
            & \frac{\partial f}{\partial x_2}(x)
            & ...
            & \frac{\partial f}{\partial x_n}(x)
        \end{bmatrix}
        $.

        If f is differentiable at every x $\in$ X, then f is differentiable on X.

        \vspace{0.5cm}

        The {\color{lblue} gradient} of f:

        \hspace{0.5cm}
        $\nabla f(x)$
        = $(\frac{\partial f}{\partial x_1}(x)
            ,...,\frac{\partial f}{\partial x_n}(x))$
        = [Df(x)]$^T$
    \end{definition}

    \vspace{0.5cm}



    \begin{definition}{Differentiability in $\mathbb{R}^n$
    $\rightarrow$ $\mathbb{R}^m$}{14cm}
        Differentiability can be extended into $\mathbb{R}^m$.

        f: X $\subset$ $\mathbb{R}^n$ $\rightarrow$ $\mathbb{R}^m$
        where f = $(f_1,...,f_m)$
        is differentiable at x $\in$ X if there is an
        A $\in$ $M_{m \times n}(\mathbb{R})$ such that for h $\in$ X:

        \hspace{0.5cm}
        $\lim_{h \rightarrow 0}$ $\frac{|f(x+h)-f(x)-Ah|}{||h||}$ = 0

        Then, the derivative of f at x is Df(x) = A =
        $
        \begin{bmatrix}
            \frac{\partial f_1}{\partial x_1}(x)
            & \frac{\partial f_1}{\partial x_2}(x)
            & ...
            & \frac{\partial f_1}{\partial x_n}(x) \\

            \frac{\partial f_2}{\partial x_1}(x)
            & \frac{\partial f_2}{\partial x_2}(x)
            & ...
            & \frac{\partial f_2}{\partial x_n}(x) \\

            \vdots & \vdots & \ddots & \vdots \\
            
            \frac{\partial f_m}{\partial x_1}(x)
            & \frac{\partial f_m}{\partial x_2}(x)
            & ...
            & \frac{\partial f_m}{\partial x_n}(x)
        \end{bmatrix}
        $.

        If f is differentiable at every x $\in$ X, then f is differentiable on X.
    \end{definition}
    
    \vspace{0.5cm}



    \begin{wtheorem}{Differentiability implies Continuity in Higher Dimensions}{14cm}
        If f: X $\subset$ $\mathbb{R}^n$ $\rightarrow$ $\mathbb{R}^m$
        is differentiable at a $\in$ X, then f is continuous at a
    \end{wtheorem}

    \begin{proof}
        Analogous to {\color{red} theorem 2.2.5}. Replace (a,b) with
        a = $(a_1,...,a_n)$.
    \end{proof}

    \newpage



    \begin{wtheorem}{Continuous partials imply differentiability
    in Higher Dimensions}{14cm}
        If f: X $\subset$ $\mathbb{R}^n$ $\rightarrow$ $\mathbb{R}^m$
        has continuous partial derivatives, $\frac{\partial f_i}{\partial x_j}$,
        at a $\in$ X for
        
        j = \{1,...,n\} and i = \{1,...,m\}, then f is differentiable at a
    \end{wtheorem}

    \begin{proof}
        Analogous to {\color{red} theorem 2.2.4}.
        Instead, h = $h_1e_1 + ... + h_ne_n$ where:

        \hspace{0.5cm}
        $\lim_{h \rightarrow 0}$ $\frac{|f(x+h)-f(x)-Ah|}{||h||}$
        = $\lim_{h \rightarrow 0}$ $\sum_{i=1}^m$
            $\frac{|f_i(x+h) - f_i(x)
                    - [\sum_{j=1}^n \frac{\partial f_i}{\partial x_j}(x)h_j]|}
                {||h||}$
        
        and add each $f_i(x+h_1e_1+...+h_ke_k)$ and apply Mean Value Theorem
        and continuity of partial derivatives analogously as performed in
        {\color{red} theorem 2.2.4}.
    \end{proof}

    \vspace{0.5cm}



    \begin{wtheorem}{Components of Differentiability}{14cm}
        f: X $\subset$ $\mathbb{R}^n$ $\rightarrow$ $\mathbb{R}^m$
        where f = $(f_1,...,f_m)$ is differentiable at a $\in$ X if and only if
        each $f_i$ is differentiable at a for i = \{1,...,m\}
    \end{wtheorem}

    \begin{proof}
        Note
        $\lim_{h \rightarrow 0}$ $\frac{|f(x+h)-f(x)-Ah|}{||h||}$
        = $\lim_{h \rightarrow 0}$ $\sum_{i=1}^m$
            $\frac{|f_i(x+h) - f_i(x)
                    - [\sum_{j=1}^n \frac{\partial f_i}{\partial x_j}(x)h_j]|}
                {||h||}$.

        If f is differentiable at a, then for any $\epsilon > 0$:

        \hspace{0.5cm}
        $\lim_{h \rightarrow 0}$ $\frac{|f(x+h)-f(x)-Ah|}{||h||}$
        $<$ $\epsilon$

        So $\lim_{h \rightarrow 0}$
                $\frac{|f_i(x+h) - f_i(x)
                - [\sum_{j=1}^n \frac{\partial f_i}{\partial x_j}(x)h_j]|}{||h||}$
            $<$ $\epsilon$
        for each i = \{1,...,m\}
        and thus, each $f_i$ is differentiable at a.

        \rule[0.1cm]{15cm}{0.01cm}

        If each $\lim_{h \rightarrow 0}$
                $\frac{|f_i(x+h) - f_i(x)
                - [\sum_{j=1}^n \frac{\partial f_i}{\partial x_j}(x)h_j]|}{||h||}$
                $<$ $\frac{\epsilon}{m}$
        for i = \{1,...,m\}, then:

        \hspace{0.5cm}
        $\lim_{h \rightarrow 0}$ $\frac{|f(x+h)-f(x)-Ah|}{||h||}$
        $<$ $\lim_{h \rightarrow 0}$ $\sum_{i=1}^m$ $\frac{\epsilon}{m}$
        = $\epsilon$

        Thus, f is differentiable at a.
    \end{proof}

    \vspace{0.5cm}

    

    \begin{ltheorem}{Properties of Differentiability}{1.5cm}
        \item For f,g: X $\subset$ $\mathbb{R}^n$ $\rightarrow$ $\mathbb{R}^m$,
            if f,g are differentiable at a $\in$ X, then:

            \hspace{0.5cm}
            f+g is differentiable at a where D(f+g)(a) = Df(a) + Dg(a)

            \begin{proof}[14cm]
                Since f,g are differentiable at a $\in$ X,
                by {\color{red} theorem 2.3.5}, then for i = \{1,...,m\}:

                \hspace{0.5cm}
                D$(f+g)_i$(a) =
                $
                \begin{bmatrix}
                    D_1(f_i+g_i)(a)
                    & D_2(f_i+g_i)(a)
                    & ...
                    & D_n(f_i+g_i)(a)
                \end{bmatrix}
                $

                \hspace{0.5cm}
                = $
                \begin{bmatrix}
                    D_1f_i(a)
                    & D_2f_i(a)
                    & ...
                    & D_nf_i(a)
                \end{bmatrix}
                + \begin{bmatrix}
                    D_1g_i(a)
                    & D_2g_i(a)
                    & ...
                    & D_ng_i(a)
                \end{bmatrix}
                $

                \hspace{0.5cm}
                = D$f_i$(a) + D$g_i$(a)
            \end{proof}

        \item For f: X $\subset$ $\mathbb{R}^n$ $\rightarrow$ $\mathbb{R}^m$,
            if f is differentiable at a $\in$ X and scalar c $\in$ $\mathbb{R}$,
            then:

            \hspace{0.5cm}
            cf is differentiable at a where D(cf)(a) = cDf(a)

            \begin{proof}[14cm]
                Since f is differentiable at a $\in$ X,
                by {\color{red} theorem 2.3.5}, then for i = \{1,...,m\}:

                \hspace{0.5cm}
                D$(cf)_i$(a) =
                $
                \begin{bmatrix}
                    D_1(cf_i)(a)
                    & D_2(cf_i)(a)
                    & ...
                    & D_n(cf_i)(a)
                \end{bmatrix}
                $

                \hspace{2.3cm}
                = $
                \begin{bmatrix}
                    cD_1f_i(a)
                    & cD_2f_i(a)
                    & ...
                    & cD_nf_i(a)
                \end{bmatrix}
                $
                = cD$f_i$(a)
            \end{proof}

            \newpage

        \item For f,g: X $\subset$ $\mathbb{R}^n$ $\rightarrow$ $\mathbb{R}$,
            if f,g are differentiable at a $\in$ X, then:

            \hspace{0.5cm}
            fg is differentiable at a where D(fg)(a) = Df(a)g(a) + f(a)Dg(a)

            \begin{proof}[14cm]
                Since f,g are differentiable at a $\in$ X:

                \hspace{0.2cm}
                D(fg)(a) =
                $
                \begin{bmatrix}
                    D_1(fg)(a)
                    & D_2(fg)(a)
                    & ...
                    & D_n(fg)(a)
                \end{bmatrix}
                $

                \hspace{0.2cm}
                = $
                \begin{bmatrix}
                    \scriptstyle D_1f(a)g(a)+f(a)D_1g(a)
                    & \scriptstyle D_2f(a)g(a)+f(a)D_2g(a)
                    & \scriptstyle ...
                    & \scriptstyle D_nf(a)g(a)+f(a)D_ng(a)
                \end{bmatrix}
                $

                \hspace{0.2cm}
                = $
                \begin{bmatrix}
                    \scriptstyle D_1f(a)g(a)
                    & \scriptstyle D_2f(a)g(a)
                    & \scriptstyle ...
                    & \scriptstyle D_nf(a)g(a)
                \end{bmatrix} +
                \begin{bmatrix}
                    \scriptstyle f(a)D_1g(a)
                    & \scriptstyle f(a)D_2g(a)
                    & \scriptstyle ...
                    & \scriptstyle f(a)D_ng(a)
                \end{bmatrix}
                $

                \hspace{0.2cm}
                = Df(a)g(a) + f(a)Dg(a)
            \end{proof}

        \item For f,g: X $\subset$ $\mathbb{R}^n$ $\rightarrow$ $\mathbb{R}$,
            if f,g are differentiable at a $\in$ X where g(a) $\not =$ 0, then:

            \hspace{0.5cm}
            $\frac{\text{f}}{\text{g}}$ is differentiable at a
            where D($\frac{\text{f}}{\text{g}}$)(a)
            = $\frac{Df(a)g(a) - f(a)Dg(a)}{[g(a)]^2}$

            \begin{proof}[14cm]
                Since f,g are differentiable at a $\in$ X:

                \hspace{0.2cm}
                D($\frac{f}{g}$)(a) =
                $
                \begin{bmatrix}
                    D_1(\frac{f}{g})(a)
                    & D_2(\frac{f}{g})(a)
                    & ...
                    & D_n(\frac{f}{g})(a)
                \end{bmatrix}
                $

                \hspace{0.2cm}
                = $
                \begin{bmatrix}
                    \scriptstyle \frac{D_1f(a)g(a)-f(a)D_1g(a)}{[g(a)]^2}
                    & \scriptstyle \frac{D_2f(a)g(a)-f(a)D_2g(a)}{[g(a)]^2}
                    & \scriptstyle ...
                    & \scriptstyle \frac{D_nf(a)g(a)-f(a)D_ng(a)}{[g(a)]^2}
                \end{bmatrix}
                $

                \hspace{0.2cm}
                = $
                \begin{bmatrix}
                    \scriptstyle \frac{D_1f(a)g(a)}{[g(a)]^2}
                    & \scriptstyle \frac{D_2f(a)g(a)}{[g(a)]^2}
                    & \scriptstyle ...
                    & \scriptstyle \frac{D_nf(a)g(a)}{[g(a)]^2}
                \end{bmatrix} -
                \begin{bmatrix}
                    \scriptstyle \frac{f(a)D_1g(a)}{[g(a)]^2}
                    & \scriptstyle \frac{f(a)D_1g(a)}{[g(a)]^2}
                    & \scriptstyle ...
                    & \scriptstyle \frac{f(a)D_1g(a)}{[g(a)]^2}
                \end{bmatrix}
                $

                \hspace{0.2cm}
                = Df(a)$\frac{g(a)}{[g(a)]^2}$ - $\frac{f(a)}{[g(a)]^2}$Dg(a)
                = $\frac{Df(a)g(a) - f(a)Dg(a)}{[g(a)]^2}$
            \end{proof}
    \end{ltheorem}

    \vspace{0.5cm}



    \begin{definition}{Partial Derivatives of Higher Orders}{14cm}
        The {\color{lblue} second order partial derivative} of f
        in respect to $x_i$:

        \hspace{0.5cm}
        $\frac{\partial^2 f}{\partial x_i^2}$
        = $\frac{\partial}{\partial x_i} (\frac{\partial f}{\partial x_i})$
        = $f_{x_ix_i}(x)$
        = $\lim_{h \rightarrow 0}$
            $\frac{f_{x_i}(x_1,...,x_i+h,...,x_n) - f_{x_i}(x_1,...,x_n)}{h}$

        \vspace{0.3cm}

        The {\color{lblue} mixed partial derivative} of f
        in respect to first $x_i$, then $x_j$:

        \hspace{0.5cm}
        $\frac{\partial^2 f}{\partial x_j \partial x_i}$
        = $\frac{\partial}{\partial x_j} (\frac{\partial f}{\partial x_i})$
        = $f_{x_ix_j}(x)$
        = $\lim_{h \rightarrow 0}$
            $\frac{f_{x_i}(x_1,...,x_j+h,...,x_n) - f_{x_i}(x_1,...,x_n)}{h}$

        \vspace{0.3cm}
        
        In general, for f: X $\subset$ $\mathbb{R}^n$ $\rightarrow$ $\mathbb{R}$,
        the k-th order partial derivative of f in respect to
        $x_{i_1},...,x_{i_k}$ in such order for k = \{1,...,n\}:

        \hspace{0.5cm}
        $\frac{\partial^k f}{\partial x_{i_k} ... \partial x_{i_1}}$
        = $\frac{\partial}{\partial x_{i_k}} ... \frac{\partial f}{\partial x_{i_1}}$
        = $f_{x_{i_1}...x_{i_k}}(x)$

        \hspace{0.5cm}
        = $\lim_{h \rightarrow 0}$
            $\frac{f_{x_{i_1}...x_{i_{k-1}}}(x_1,...,x_{i_k}+h,...,x_n)
                    - f_{x_{i_1}...x_{i_{k-1}}}(x_1,...,x_n)}{h}$
    \end{definition}

    \vspace{0.5cm}



    \begin{definition}{Smoothness}{14cm}
        For k = \{1,...,n\},
        f: X $\subset$ $\mathbb{R}^n$ $\rightarrow$ $\mathbb{R}$ is $C^k$
        if all partial derivatives of order 1 to k exist and are continuous
        on X.

        \vspace{0.3cm}

        If f has continuous partial derivatives of all order, then
        f is {\color{lblue} smooth} (i.e $C^{\infty}$).

        \vspace{0.3cm}

        For f: X $\subset$ $\mathbb{R}^n$ $\rightarrow$ $\mathbb{R}^m$
        where f = $(f_1,...,f_m)$, then f is $C^k$ if each $f_i$ is $C^k$
        
        for i = \{1,...,m\}.
    \end{definition}

    \newpage



    \begin{wtheorem}{Clairaut's Theorem}{14cm}
        If f: X $\subset$ $\mathbb{R}^n$ $\rightarrow$ $\mathbb{R}$
        is $C^k$, then:

        \hspace{0.5cm}
        $\frac{\partial^k f}{\partial x_{i_1} ... \partial x_{i_k}}$
        = $\frac{\partial^k f}{\partial x_{j_1} ... \partial x_{j_k}}$
    \end{wtheorem}

    \begin{proof}
        If the claim holds true for $C^2$, then replace f with
        $f_{x_{i_p}}$ for any p = \{1,...,k\} and since f is $C^k$,
        then $f_{x_{i_p}}$ is $C^{k-1}$ and apply the theorem again.
        Repeating the process k times, the result holds true by induction.
        Now the proof for $C^2$:

        \vspace{0.3cm}

        Since f is $C^2$, then $f_x,f_y,f_{xy},f_{yx}$
        exist and are continuous.

        Let d(x,y) = $f(x+h_1,y+h_2) - {\color{red} f(x,y+h_2)}
                        - {\color{lblue} f(x+h_1,y)} + f(x,y)$.

        Since $f_x$ exist, then by the Mean Value Theorem,
        there is a $t_1$ $\in$ $(0,h_1)$ where:

        \hspace{0.5cm}
        d(x,y) = $h_1*(f_x(x+t_1,y+h_2) - f_x(x+t_1,y))$

        Since $f_y$ exist, then by the Mean Value Theorem,
        there is a $t_2$ $\in$ $(0,h_2)$ where:

        \hspace{0.5cm}
        d(x,y) = $h_1*h_2*f_{xy}(x+t_1,y+t_2)$

        Since $f_{xy}$ is continuous, then since
        $(t_1,t_2) \rightarrow (0,0)$ as $(h_1,h_2) \rightarrow (0,0)$:

        \hspace{0.5cm}
        $f_{xy}(x,y)$
        = $\lim_{(h_1,h_2) \rightarrow (0,0)}$ $f_{xy}(x+h_1,x+h_2)$

        \hspace{2.15cm}
        = $\lim_{(h_1,h_2) \rightarrow (0,0)}$ $f_{xy}(x+t_1,x+t_2)$
        = $\lim_{(h_1,h_2) \rightarrow (0,0)}$ $\frac{d(x,y)}{h_1h_2}$

        Rearrange d(x,y) = $f(x+h_1,y+h_2) - {\color{lblue} f(x+h_1,y)}
                           - {\color{red} f(x,y+h_2)} + f(x,y)$.

        Since $f_y$ exist, by the Mean Value Theorem,
        there is a $s_2$ $\in$ $(0,h_2)$ where:

        \hspace{0.5cm}
        d(x,y) = $h_2 * (f_y(x+h_1,y+s_2) - f_y(x,y+s_2))$

        Since $f_x$ exist, by the Mean Value Theorem,
        there is a $s_1$ $\in$ $(0,h_1)$ where:

        \hspace{0.5cm}
        d(x,y) = $h_2 * h_1 * f_{yx}(x+s_1,y+s_2)$

        Since $f_{yx}$ is continuous, then since
        $(s_1,s_2) \rightarrow (0,0)$ as $(h_1,h_2) \rightarrow (0,0)$:

        \hspace{0.5cm}
        $f_{yx}(x,y)$
        = $\lim_{(h_1,h_2) \rightarrow (0,0)}$ $f_{yx}(x+h_1,x+h_2)$

        \hspace{2.15cm}
        = $\lim_{(h_1,h_2) \rightarrow (0,0)}$ $f_{xy}(x+s_1,x+s_2)$
        = $\lim_{(h_1,h_2) \rightarrow (0,0)}$ $\frac{d(x,y)}{h_1h_2}$

        Thus, $f_{xy}(x,y)$ = $f_{yx}(x,y)$.
    \end{proof}

    \vspace{0.5cm}



    \begin{wtheorem}{Chain Rule}{14cm}
        Let f: X $\subset$ $\mathbb{R}^n$ $\rightarrow$ $\mathbb{R}^m$
        be differentiable at $x_0$ $\in$ X and g: f(X) $\subset$ Y $\subset$
        $\mathbb{R}^m$ $\rightarrow$ $\mathbb{R}^k$ be differentiable at f($x_0$).
        
        Then g $\circ$ f = g(f): X $\subset$ $\mathbb{R}^n$ $\rightarrow$
        $\mathbb{R}^k$ is differentiable at $x_0$ such that:

        \hspace{0.5cm}
        D[g(f($x_0$))] = Dg(f($x_0$)) Df($x_0$) 
    \end{wtheorem}

    \begin{proof}
        Since f is differentiable at $x_0$ and g is differentiable at f($x_0$),
        then there is a A = Df($x_0$) and B = Dg(f($x_0$)) such that:

        \hspace{0.5cm}
        f($x_0$+h) - f($x_0$)
        = Ah + $r_A(h)$
        \hspace{2cm}
        where $\lim_{h \rightarrow 0}$ $\frac{|r_A(h)|}{|h|}$ = 0

        \hspace{0.5cm}
        g(f($x_0$)+k) - g(f($x_0$))
        = Bk + $r_B(k)$
        \hspace{1cm}
        where $\lim_{k \rightarrow 0}$ $\frac{|r_B(k)|}{|k|}$ = 0

        Let k = f($x_0$+h) - f($x_0$). Thus:

        \hspace{0.5cm}
        g(f($x_0$+h)) - g(f($x_0$)) - BAh

        \hspace{0.5cm}
        = g(f($x_0$)+k) - g(f($x_0$)) - BAh
        = Bk + $r_B(k)$ - BAh
        = B(k - Ah) + $r_B(k)$

        \hspace{0.5cm}
        = B(f($x_0$+h) - f($x_0$) - Ah) + $r_B(k)$
        = B$r_A(h)$ + $r_B(k)$

        Since f is differentiable at $x_0$, then f is continuous at $x_0$
        and thus, $\lim_{h \rightarrow 0}$ k = 0.

        Since $\lim_{h \rightarrow 0}$ $\frac{|r_A(h)|}{|h|}$ = 0
        and $\lim_{k \rightarrow 0}$ $\frac{|r_A(k)|}{|k|}$ = 0, then:

        \hspace{0.5cm}
        $\lim_{h \rightarrow 0}$ $\frac{| g(f(x_0+h)) - g(f(x_0)) - BAh|}{|h|}$
        $\leq$ $\lim_{h \rightarrow 0}$ ($||B|| \frac{|r_A(h)|}{|h|}$
                + $\frac{|r_B(k)|}{|h|}$)
        = $0+0$ = 0

        Thus, D[g(f($x_0$))] = BA = Dg(f($x_0$)) Df($x_0$).
    \end{proof}

    \newpage



    \begin{wtheorem}{Relationship between rectangular and polar partials}{14cm}
        For (x,y) = $(r\cos(\theta),r\sin(\theta))$:
        
        \hspace{0.5cm}
        $\frac{\partial}{\partial r}$
        = $\cos(\theta) \frac{\partial}{\partial x}
            + \sin(\theta) \frac{\partial}{\partial y}$

        \hspace{0.5cm}
        $\frac{\partial}{\partial \theta}$
        = $-r \sin(\theta) \frac{\partial}{\partial x}
            + r \cos(\theta) \frac{\partial}{\partial y}$

        Thus:

        \hspace{0.5cm}
        $\frac{\partial}{\partial x}$
        = $\cos(\theta) \frac{\partial}{\partial r}
            - \frac{\sin(\theta)}{r} \frac{\partial}{\partial \theta}$

        \hspace{0.5cm}
        $\frac{\partial}{\partial y}$
        = $\sin(\theta) \frac{\partial}{\partial r}
            + \frac{\cos(\theta)}{r} \frac{\partial}{\partial \theta}$
    \end{wtheorem}

    \begin{proof}
        Let z = g($r,\theta$). Then let z = f(x,y)
        such that $(r\cos(\theta),r\sin(\theta))$ = (x,y).
        
        By {\color{red} theorem 2.3.10}:

        \hspace{0.2cm}
        D[g($r,\theta$)] = D(f($x,y$)) D(x($r,\theta$),y($r,\theta$))
        
        \hspace{0.2cm}
        $
        \begin{bmatrix}
            \frac{\partial z}{\partial r} & \frac{\partial z}{\partial \theta} 
        \end{bmatrix}
        $
        =
        $
        \begin{bmatrix}
            \frac{\partial z}{\partial x} & \frac{\partial z}{\partial y}
        \end{bmatrix}
        \begin{bmatrix}
            \frac{\partial x}{\partial r} & \frac{\partial x}{\partial \theta} \\
            \frac{\partial y}{\partial r} & \frac{\partial y}{\partial \theta}
        \end{bmatrix}
        $
        =
        $
        \begin{bmatrix}
            \frac{\partial f}{\partial x} & \frac{\partial f}{\partial y}
        \end{bmatrix}
        \begin{bmatrix}
            \scriptstyle \cos(\theta) & \scriptstyle -r\sin(\theta) \\
            \scriptstyle \sin(\theta) & \scriptstyle r\cos(\theta)
        \end{bmatrix}
        $
        =
        $
        \begin{bmatrix}
            \scriptstyle \cos(\theta) \frac{z\partial}{\partial x}
                + \sin(\theta) \frac{z\partial}{\partial y} \\
            \scriptstyle -r \sin(\theta) \frac{z\partial}{\partial x}
                + r \cos(\theta) \frac{z\partial}{\partial y} \\
        \end{bmatrix}
        $

        Thus:

        \hspace{0.5cm}
        $\frac{\partial}{\partial r}$
        = $\cos(\theta) \frac{\partial}{\partial x}
            + \sin(\theta) \frac{\partial}{\partial y}$

        \hspace{0.5cm}
        $\frac{\partial}{\partial \theta}$
        = $-r \sin(\theta) \frac{\partial}{\partial x}
            + r \cos(\theta) \frac{\partial}{\partial y}$

        Then:

        \hspace{0.5cm}
        $-r \cos(\theta) \frac{\partial }{\partial r}
            + \sin(\theta) \frac{\partial}{\partial \theta}$
        = $-r\cos^2(\theta)\frac{\partial }{\partial x}
            -r\sin^2(\theta)\frac{\partial }{\partial x}$
        = $-r \frac{\partial }{\partial x}$

        \hspace{0.5cm}
        $r \sin(\theta) \frac{\partial }{\partial r}
            + \cos(\theta) \frac{\partial}{\partial \theta}$
        = $r\sin^2(\theta)\frac{\partial }{\partial y}
            + r\cos^2(\theta)\frac{\partial }{\partial y}$
        = $r \frac{\partial }{\partial y}$
    \end{proof}

    \vspace{0.5cm}





\subsection{ Directional Derivative }

    \begin{definition}{Directional Derivative}{14cm}
        Let f: X $\subset$ $\mathbb{R}^n$ $\rightarrow$ $\mathbb{R}$
        be differentiable at a $\in$ X.
        Then the {\color{lblue} directional derivative} of f at a
        in the direction of vector v $\in$ $\mathbb{R}^n$:

        \hspace{0.5cm}
        $D_vf(a)$ = $\lim_{h \rightarrow 0}$ $\frac{f(a+hv) - f(a)}{||hv||}$
    \end{definition}

    \vspace{0.5cm}



    \begin{wtheorem}{Relationship between Directional Derivative and Gradient}{14cm}
        Let f: X $\subset$ $\mathbb{R}^n$ $\rightarrow$ $\mathbb{R}$
        be differentiable at a $\in$ X.
        Then the directional derivative of f at a
        in the direction of vector v $\in$ $\mathbb{R}^n$:

        \hspace{0.5cm}
        $D_vf(a)$ = $\nabla f(a)$ $\cdot$ $\frac{v}{||v||}$

        If v is a unit vector, then $D_vf(a)$ = $\nabla f(a)$ $\cdot$ v.
    \end{wtheorem}

    \begin{proof}
        Let y(t) = a+tv for t $\in$ $(-\infty,\infty)$.
        Then by {\color{red} theorem 2.3.10}:

        \hspace{0.5cm}
        $D_vf(a)$
        = $\lim_{h \rightarrow 0}$ $\frac{f(a+hv) - f(a)}{||hv||}$
        = $\lim_{h \rightarrow 0}$ $\frac{f(y(h)) - f(y(0))}{|h|} \frac{1}{||v||}$
        = Df(y(0)) Dy(0) $\frac{1}{||v||}$

        \hspace{1.95cm}
        = Df(a)) $\frac{v}{||v||}$
        = [Df(a))]$^T$ $\cdot$ $\frac{v}{||v||}$
        = $\nabla f(a)$ $\cdot$ $\frac{v}{||v||}$
    \end{proof}

    \newpage



    \begin{wtheorem}{Direction of Steepest Ascent}{14cm}
        The directional derivative
        $D_vf(a)$ = $\nabla f(a)$ $\cdot$ $\frac{v}{||v||}$ is:
        
        \hspace{0.3cm}
        Maximized when v is in the same direction as $\nabla f(a)$
        with value $||\nabla f(a)||$

        \hspace{0.3cm}
        Minimized when v is in the opposite direction of $\nabla f(a)$
        with value $-||\nabla f(a)||$
    \end{wtheorem}

    \begin{proof}
        By {\color{red} theorem 1.2.3},
        $D_vf(a)$
        = $\nabla f(a)$ $\cdot$ $\frac{v}{||v||}$
        = $||\nabla f(a)||$ $||\frac{v}{||v||}||$ cos($\theta$)
        = $||\nabla f(a)||$ cos($\theta$).
        
        where $\theta$ $\in$ $[0,\pi]$ is the angle between
        $\nabla f(a)$ and $|\frac{v}{||v||}|$.

        Since $D_vf(a)$ is maximized at $||\nabla f(a)||$ when $\theta$ = 0,
        then $\nabla f(a)$ and v points in the same direction.
        Also, $D_vf(a)$ is minimized at $-||\nabla f(a)||$ when $\theta$ = $\pi$,
        then $\nabla f(a)$ and v points in opposite directions.
    \end{proof}

    \vspace{0.5cm}



    \begin{wtheorem}{Gradient is orthogonal to the surface}{14cm}
        If f: X $\subset$ $\mathbb{R}^n$ $\rightarrow$ $\mathbb{R}$
        is $C^1$, then for any $x_0$ where f($x_0$) = c for constant
        c $\in$ $\mathbb{R}$, then $\nabla f(x_0)$ is orthogonal
        to the surface f(x) = c at $x_0$.
    \end{wtheorem}

    \begin{proof}
        For surface f(x) = c, let curve C(t) = $(x_1(t),...,x_n(t))$
        where C(0) = $x_0$ be defined such that f(C(t)) = c.
        Thus:

        \hspace{0.5cm}
        $\frac{d}{dt}$ f(C(t))
        = $\frac{d}{dt}$ c = 0

        Let v be the tangent vector to C(t) at $x_0$.
        Then by {\color{red} theorem 2.3.10}, for t = 0:

        \hspace{0.5cm}
        $\frac{d}{dt}$ f(C(0))
        = Df(C(0)) C'(0)
        = $\nabla f(C(0))$ $\cdot$ C'(0)
        = $\nabla f(x_0)$ $\cdot$ v

        Since $\nabla f(x_0)$ $\cdot$ v = 0
        where v is tangent to C(t) which lies on surface f(x) = c and thus,
        is tangent is f(x) = c, then $\nabla f(x_0)$ is orthogonal to f(x) = c
        at $x_0$.
    \end{proof}















