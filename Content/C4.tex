\newpage

\section[Day 4: Integration]{ Integration }

\subsection{ Double Integral }

    \begin{definition}{Riemann Sum in $\mathbb{R}^2$}{14cm}
        For region R = [a,b] $\times$ [c,d], a {\color{lblue} partition}, P,
        consist:
        
        \hspace{0.5cm}
        \{$x_0,...,x_n$\} $\in$ [a,b] such that each $x_i < x_{i+1}$
        and $\Delta x_i$ = $x_i - x_{i-1}$
        
        \hspace{0.5cm}
        \{$y_0,...,y_n$\} $\in$ [c,d] such that each $y_i < y_{i+1}$
        and $\Delta y_i$ = $y_i - y_{i-1}$

        Let $||P||$ = max($\Delta x_i$,$\Delta y_j$) for i,j = \{1,...,n\}.
        Then for each $[x_{i-1},x_i]$ $\times$ $[y_{j-1},y_j]$,
        choose one point $c_{ij}$.
        Let {\color{lblue} sample points}, C, consist of all $c_{ij}$.

        Then the {\color{lblue} Riemann sum} of f(x,y) for partition P,
        sample points C:

        \hspace{0.5cm}
        R(f,P,C) = $\sum_{i,j=1}^n$ $f(c_{ij}) \Delta x_i \Delta y_j$
    \end{definition}

    \vspace{0.5cm}



    \begin{definition}{Double Integral}{14cm}
        f: R $\subset$ $\mathbb{R}^2$ $\rightarrow$ $\mathbb{R}$
        is {\color{lblue} integrable} if:

        \hspace{0.5cm}
        $\int \int_R$ $f(x,y)$ dA
        = $\int \int_R$ $f(x,y)$ dxdy
        = $\underset{||P|| \rightarrow 0}{\lim}$
            $\sum_{i,j=1}^n$ $f(c_{ij}) \Delta x_i \Delta y_j$

        exist. Then, $\int \int_R$ $f(x,y)$ dA
        is the {\color{lblue} double integral} of f on R.
    \end{definition}

    \vspace{0.5cm}



    \begin{wtheorem}{Continuous functions in $\mathbb{R}^2$ are integrable}{14cm}
        If f: R $\subset$ $\mathbb{R}^2$ $\rightarrow$ $\mathbb{R}$
        is continuous on region R, then $\int \int_R$ $f(x,y)$ dA exists.
    \end{wtheorem}

    \vspace{0.5cm}



    \begin{wtheorem}{Piecewise continuous functions in $\mathbb{R}^2$
    are integrable}{14cm}
        Set X $\subset$ $\mathbb{R}^2$ has {\color{lblue} measure zero}
        if for any $\epsilon > 0$ and every $x_i$ $\in$ X, there are
        $[a_i,b_i] \times [c_i,d_i]$ where $x_i$ $\in$
        $[a_i,b_i] \times [c_i,d_i]$ such that:

        \hspace{0.5cm}
        $\sum_{x_i \in X}$ $(b_i - a_i)*(d_i - c_i)$ $<$ $\epsilon$

        \vspace{0.3cm}
        
        If f: R $\subset$ $\mathbb{R}^2$ $\rightarrow$ $\mathbb{R}$
        is bounded and the set S of discontinuities of f on R
        has measure zero, then $\int \int_R$ $f(x,y)$ dA exists.
    \end{wtheorem}

    \vspace{0.5cm}



    \begin{wtheorem}{Fubini's Theorem in $\mathbb{R}^2$:
    Double Integrals $\Rightarrow$ Iterated Integrals}{14cm}
        Let f: R = [a,b] $\times$ [c,d] $\subset$ $\mathbb{R}^2$
        $\rightarrow$ $\mathbb{R}$ be bounded and the set S of
        discontinuities of f on R have measure zero.
        If every line parallel to x=0 or y=0 contains finitely
        many points on S, then:

        \hspace{0.5cm}
        $\int \int_R$ $f(x,y)$ dA
        = $\int_c^d \int_a^b$ $f(x,y)$ dxdy
        = $\int_a^b \int_c^d$ $f(x,y)$ dydx 
    \end{wtheorem}

    \vspace{0.5cm}



    \begin{ltheorem}{Properties of the Integral in $\mathbb{R}^2$}{1.5cm}
        \item If f,g are integrable on R, then f+g is integrable on R where:
        
            \hspace{0.5cm}
            $\int \int_R$ f+g dA
            = $\int \int_R$ f dA + $\int \int_R$ g dA

        \item If f is integrable on R and scalar c $\in$ $\mathbb{R}$,
            then cf is integrable on R where:
        
            \hspace{0.5cm}
            $\int \int_R$ cf dA
            = c $\int \int_R$ f dA

        \item If f,g are integrable on R where f(x,y) $\leq$ g(x,y) for all
            (x,y) $\in$ R:
        
            \hspace{0.5cm}
            $\int \int_R$ f dA
            $\leq$ $\int \int_R$ g dA

        \item If f is integrable on R, then $|f|$ is integrable on R where:
        
            \hspace{0.5cm}
            $|$$\int \int_R$ f dA$|$
            $\leq$ $\int \int_R$ $|f|$ dA
    \end{ltheorem}

    \newpage




    \begin{definition}{Elementary Region and the Extension of f
    in $\mathbb{R}^2$}{14cm}
        Set D $\subset$ $\mathbb{R}^2$ is an {\color{lblue} elementary region} if:

        \begin{enumerate}[label=(\alph*), leftmargin=1cm, itemsep=0.1cm]
            \item {\color{red} Type 1}

                For x $\in$ [a,b], there are continuous $y_1(x)$,$y_2(x)$ where
                $y_1(x)$ $\leq$ y $\leq$ $y_2(x)$.

                \hspace{0.5cm}
                D = \{ (x,y) $|$ x $\in$ [a,b] , y $\in$ [$y_1(x)$,$y_2(x)$] \}

            \item {\color{red} Type 2}

                For y $\in$ [c,d], there are continuous $x_1(y)$,$x_2(y)$ where
                $x_1(y)$ $\leq$ x $\leq$ $x_2(y)$.

                \hspace{0.5cm}
                D = \{ (x,y) $|$ x $\in$ [$x_1(y)$,$x_2(y)$] , y $\in$ [c,d] \}

            \item {\color{red} Type 3}
            
                Both Type 1 and Type 2
        \end{enumerate}

        \vspace{0.3cm}

        If f: D $\rightarrow$ $\mathbb{R}^2$ $\rightarrow$ $\mathbb{R}$
        is continuous and D is an elementary region, then
        the extension of f, $f^{ext}$:

        \hspace{0.5cm}
        $f^{ext}(x,y)$ =
        $
        \begin{cases}
            f(x,y) & (x,y) \in D \\
            0 & (x,y) \not \in D
        \end{cases}
        $

        Since f is continuous and thus, integrable and the discontinuities of
        $f^{ext}$ exist only at the boundary of D which is a curve and thus,
        have measure zero, then $f^{ext}$ is integrable.
    \end{definition}

    \vspace{0.5cm}



    \begin{wtheorem}{Double Integral over a General Region}{14cm}
        If f: D $\rightarrow$ $\mathbb{R}^2$ $\rightarrow$ $\mathbb{R}$
        is continuous and D is an elementary region, then
        for any region R = [a,b] $\times$ [c,d] such that D $\subset$ R:

        \hspace{0.5cm}
        $\int \int_D$ f(x,y) dA
        = $\int \int_R$ $f^{ext}(x,y)$ dA

        Then if:

        \begin{enumerate}[label=(\alph*), leftmargin=1cm, itemsep=0.1cm]
            \item D is a Type 1 elementary region
            
                \hspace{0.5cm}
                $\int \int_D$ f(x,y) dA
                = $\int_a^b \int_{y_1(x)}^{y_2(x)}$ f(x,y) dydx

            \item D is a Type 2 elementary region
            
                \hspace{0.5cm}
                $\int \int_D$ f(x,y) dA
                = $\int_c^d \int_{x_1(y)}^{x_2(y)}$ f(x,y) dxdy
        \end{enumerate}
    \end{wtheorem}

    \begin{proof}
        Suppose D is a Type 1 elementary region.

        Let region R = [a,b] $\times$ [c,d] be such that
        c $\leq$ $y_1(x)$ $\leq$ y $\leq$ $y_2(x)$ $\leq$ d.

        Since $f^{ext}(x,y)$ = 0 for any
        (x,y) $\not \in$ D, then by {\color{red} theorem 4.1.5}:

        \hspace{0.2cm}
        $\int \int_D$ f(x,y) dA
        = $\int \int_R$ $f^{ext}(x,y)$ dA
        = $\int_a^b \int_c^d$ $f^{ext}(x,y)$ dydx
        = $\int_a^b \int_{y_1(x)}^{y_2(x)}$ f(x,y) dydx

        Proof is analogous for a Type 2 elementary region.
    \end{proof}

    \vspace{0.5cm}



    \begin{example}
        Let f(x,y) = x+y.
        Find $\int \int_D$ f(x,y) dA for (x,y) $\in$ Triangle((0,0),(3,0),(0,4)).
    \end{example}

    \begin{tbox}
        $\int \int_D$ f(x,y) dA
        = $\int_0^3 \int_0^{\frac{4}{3}x}$ x+y dydx
        = $\int_0^3$ $[xy+\frac{1}{2}y^2]_0^{\frac{4}{3}x}$ dx
        = $\int_0^3$ $\frac{20}{9}x^2$ dx
        = 20

        \hspace{2.65cm}
        = $\int_0^4 \int_{\frac{3}{4}y}^3$ x+y dxdy
        = $\int_0^4$ $[\frac{1}{2}x^2+xy]_{\frac{3}{4}y}^3$ dy
        = $\int_0^4$ $\frac{9}{2} + 3y - \frac{33}{32}y^2$ dy
        = 20
    \end{tbox}

    \vspace{0.5cm}



    \begin{definition}{Area}{14cm}
        The {\color{lblue} area} of a region R $\subset$ $\mathbb{R}^2$:

        \hspace{0.5cm}
        A = $\int \int_R$ 1 dA
    \end{definition}

    \newpage





\subsection{ Triple Integral }

        \begin{definition}{Riemann Sum in $\mathbb{R}^3$}{14cm}
        For region R = [a,b] $\times$ [c,d] $\times$ [e,f],
        a {\color{lblue} partition}, P, consist:
        
        \hspace{0.5cm}
        \{$x_0,...,x_n$\} $\in$ [a,b] such that each $x_i < x_{i+1}$
        and $\Delta x_i$ = $x_i - x_{i-1}$
        
        \hspace{0.5cm}
        \{$y_0,...,y_n$\} $\in$ [c,d] such that each $y_i < y_{i+1}$
        and $\Delta y_i$ = $y_i - y_{i-1}$

        \hspace{0.5cm}
        \{$z_0,...,z_n$\} $\in$ [e,f] such that each $z_i < z_{i+1}$
        and $\Delta z_i$ = $z_i - z_{i-1}$

        Let $||P||$ = max($\Delta x_i,\Delta y_j,\Delta z_k$)
        for i,j,k = \{1,...,n\}.
        Then for each $[x_{i-1},x_i]$ $\times$ $[y_{j-1},y_j]$
        $\times$ $[z_{k-1},z_k]$,
        choose one point $c_{ijk}$.
        Let {\color{lblue} sample points}, C, consist of all $c_{ijk}$.
        Then the {\color{lblue} Riemann sum} of f(x,y,z) for partition P,
        sample points C:

        \hspace{0.5cm}
        R(f,P,C) = $\sum_{i,j,k=1}^n$ $f(c_{ijk}) \Delta x_i \Delta y_j \Delta z_k$
    \end{definition}

    \vspace{0.5cm}



    \begin{definition}{Triple Integral}{14cm}
        f: R $\subset$ $\mathbb{R}^3$ $\rightarrow$ $\mathbb{R}$
        is {\color{lblue} integrable} if:

        \hspace{0.5cm}
        $\int \int \int_R f(x,y,z)$ dV
        = $\int \int \int_R f(x,y,z)$ dxdydz

        \hspace{4.15cm}
        = $\underset{||P|| \rightarrow 0}{\lim}$
            $\sum_{i,j=1}^n f(c_{ijk}) \Delta x_i \Delta y_j \Delta z_k$

        exist. Then, $\int \int \int_R$ $f(x,y,z)$ dV
        is the {\color{lblue} triple integral} of f on R.
    \end{definition}

    \vspace{0.5cm}



    \begin{wtheorem}{Continuous functions in $\mathbb{R}^3$ are integrable}{14cm}
        If f: R $\subset$ $\mathbb{R}^3$ $\rightarrow$ $\mathbb{R}$
        is continuous on region R, then $\int \int \int_R$ $f(x,y,z)$ dV exists.
    \end{wtheorem}

    \vspace{0.5cm}



    \begin{wtheorem}{Piecewise continuous functions in $\mathbb{R}^3$
    are integrable}{14cm}
        Set X $\subset$ $\mathbb{R}^3$ has measure zero
        if for any $\epsilon > 0$ and every $x_i$ $\in$ X, there are
        $[a_i,b_i] \times [c_i,d_i] \times [e_i,f_i]$ where $x_i$ $\in$
        $[a_i,b_i] \times [c_i,d_i] \times [e_i,f_i]$ such that:

        \hspace{0.5cm}
        $\sum_{x_i \in X}$ $(b_i - a_i)*(d_i - c_i)*(f_i - e_i)$ $<$ $\epsilon$

        \vspace{0.3cm}
        
        If f: R $\subset$ $\mathbb{R}^3$ $\rightarrow$ $\mathbb{R}$
        is bounded and the set S of discontinuities of f on R
        has measure zero, then $\int \int \int_R$ $f(x,y,z)$ dV exists.
    \end{wtheorem}

    \vspace{0.5cm}



    \begin{wtheorem}{Fubini's Theorem in $\mathbb{R}^3$:
    Triple Integrals $\Rightarrow$ Iterated Integrals}{14cm}
        Let f: R = [a,b] $\times$ [c,d] $\times$ [e,f] $\subset$ $\mathbb{R}^3$
        $\rightarrow$ $\mathbb{R}$ be bounded and the set S of
        discontinuities of f on R have measure zero.
        If every line parallel to x=0, y=0, or z=0 contains finitely
        many points on S, then:

        \hspace{0.2cm}
        $\int \int \int_R$ $f(x,y,z)$ dV
        = $\int_e^f \int_c^d \int_a^b$ $f(x,y,z)$ dxdydz
        = $\int_c^d \int_e^f \int_a^b$ $f(x,y,z)$ dxdzdy
        
        \hspace{3.75cm}
        = $\int_e^f \int_a^b \int_c^d$ $f(x,y,z)$ dydxdz
        = $\int_a^b \int_e^f \int_c^d$ $f(x,y,z)$ dydzdx

        \hspace{3.75cm}
        = $\int_c^d \int_a^b \int_e^f$ $f(x,y,z)$ dzdxdy
        = $\int_a^b \int_c^d \int_e^f$ $f(x,y,z)$ dzdydx
    \end{wtheorem}

    \vspace{0.5cm}



    \begin{ltheorem}{Properties of the Integral in $\mathbb{R}^3$}{1.5cm}
        \item If f,g are integrable on R, then f+g is integrable on R where:
        
            \hspace{0.5cm}
            $\int \int \int_R$ f+g dV
            = $\int \int \int_R$ f dV + $\int \int \int_R$ g dV

        \item If f is integrable on R and scalar c $\in$ $\mathbb{R}$,
            then cf is integrable on R where:
        
            \hspace{0.5cm}
            $\int \int \int_R$ cf dV
            = c $\int \int \int_R$ f dV

        \item If f,g are integrable on R where f(x,y) $\leq$ g(x,y) for all
            (x,y) $\in$ R:
        
            \hspace{0.5cm}
            $\int \int \int_R$ f dV
            $\leq$ $\int \int \int_R$ g dV

        \item If f is integrable on R, then $|f|$ is integrable on R where:
        
            \hspace{0.5cm}
            $|$$\int \int \int_R$ f dV$|$
            $\leq$ $\int \int \int_R$ $|f|$ dV
    \end{ltheorem}

    \newpage




    \begin{definition}{Elementary Region and the Extension of f
    in $\mathbb{R}^3$}{14cm}
        Set D $\subset$ $\mathbb{R}^3$ is an elementary region if:

        \begin{enumerate}[label=(\alph*), leftmargin=1cm, itemsep=0.1cm]
            \item {\color{red} Type 1}

                For x $\in$ [a,b], there are continuous $y_1(x)$,$y_2(x)$ where
                $y_1(x)$ $\leq$ y $\leq$ $y_2(x)$
                and continuous $z_1(x,y)$,$z_2(x,y)$ where
                $z_1(x,y)$ $\leq$ z $\leq$ $z_2(x,y)$.

                \hspace{0.5cm}
                D = \{ (x,y,z) $|$ x $\in$ [a,b] , y $\in$ [$y_1(x)$,$y_2(x)$] ,
                                z $\in$ [$z_1(x,y)$,$z_2(x,y)$] \}

                OR

                For x $\in$ [a,b], there are continuous $z_1(x)$,$z_2(x)$ where
                $z_1(x)$ $\leq$ z $\leq$ $z_2(x)$
                and continuous $y_1(x,z)$,$y_2(x,z)$ where
                $y_1(x,z)$ $\leq$ y $\leq$ $y_2(x,z)$.

                \hspace{0.5cm}
                D = \{ (x,y,z) $|$ x $\in$ [a,b] , y $\in$ [$y_1(x,z)$,$y_2(x,z)$],
                                    z $\in$ [$z_1(x)$,$z_2(x)$] \}

            \item {\color{red} Type 2}

                For y $\in$ [c,d], there are continuous $x_1(y)$,$x_2(y)$ where
                $x_1(y)$ $\leq$ x $\leq$ $x_2(y)$
                and continuous $z_1(x,y)$,$z_2(x,y)$ where
                $z_1(x,y)$ $\leq$ z $\leq$ $z_2(x,y)$.

                \hspace{0.5cm}
                D = \{ (x,y,z) $|$ x $\in$ [$x_1(y)$,$x_2(y)$] , y $\in$ [c,d] ,
                                z $\in$ [$z_1(x,y)$,$z_2(x,y)$] \}

                OR

                For y $\in$ [c,d], there are continuous $z_1(y)$,$z_2(y)$ where
                $z_1(y)$ $\leq$ z $\leq$ $z_2(y)$
                and continuous $x_1(y,z)$,$x_2(y,z)$ where
                $x_1(y,z)$ $\leq$ x $\leq$ $x_2(y,z)$.

                \hspace{0.5cm}
                D = \{ (x,y,z) $|$ x $\in$ [$x_1(y,z)$,$x_2(y,z)$] , y $\in$ [c,d] ,
                                z $\in$ [$z_1(y)$,$z_2(y)$] \}

            \item {\color{red} Type 3}
            
                For z $\in$ [e,f], there are continuous $x_1(z)$,$x_2(z)$ where
                $x_1(z)$ $\leq$ x $\leq$ $x_2(z)$
                and continuous $y_1(x,z)$,$y_2(x,z)$ where
                $y_1(x,z)$ $\leq$ y $\leq$ $y_2(x,z)$.

                \hspace{0.5cm}
                D = \{ (x,y,z) $|$ x $\in$ [$x_1(z)$,$x_2(z)$] , 
                                y $\in$ [$y_1(x,z)$,$y_2(x,z)$] ,
                                z $\in$ [e,f] \}

                OR

                For z $\in$ [e,f], there are continuous $y_1(z)$,$y_2(z)$ where
                $y_1(z)$ $\leq$ y $\leq$ $y_2(z)$
                and continuous $x_1(y,z)$,$x_2(y,z)$ where
                $x_1(y,z)$ $\leq$ x $\leq$ $x_2(y,z)$.

                \hspace{0.5cm}
                D = \{ (x,y,z) $|$ x $\in$ [$x_1(y,z)$,$x_2(y,z)$] , 
                                y $\in$ [$y_1(z)$,$y_2(z)$] ,
                                z $\in$ [e,f] \}

            \item {\color{red} Type 4}
            
                All Type 1, 2, and 3
        \end{enumerate}

        \vspace{0.3cm}

        If f: D $\rightarrow$ $\mathbb{R}^3$ $\rightarrow$ $\mathbb{R}$
        is continuous and D is an elementary region, then
        the extension of f, $f^{ext}$:

        \hspace{0.5cm}
        $f^{ext}(x,y,z)$ =
        $
        \begin{cases}
            f(x,y,z) & (x,y,z) \in D \\
            0 & (x,y,z) \not \in D
        \end{cases}
        $

        Since f is continuous and thus, integrable and the discontinuities of
        $f^{ext}$ exist only at the boundary of D which is a 2d surface and thus,
        have measure zero, then $f^{ext}$ is integrable.
    \end{definition}

    \newpage



    \begin{wtheorem}{Triple Integral over a General Region}{14cm}
        If f: D $\rightarrow$ $\mathbb{R}^3$ $\rightarrow$ $\mathbb{R}$
        is continuous and D is an elementary region, then
        for any region R = [a,b] $\times$ [c,d] $\times$ [e,f]
        such that D $\subset$ R:

        \hspace{0.5cm}
        $\int \int \int_D$ f(x,y,z) dV
        = $\int \int \int_R$ $f^{ext}(x,y,z)$ dV

        Then if:

        \begin{enumerate}[label=(\alph*), leftmargin=1cm, itemsep=0.1cm]
            \item D is a Type 1 elementary region
            
                \hspace{0.5cm}
                $\int \int \int_D$ f(x,y,z) dV
                = $\int_a^b \int_{y_1(x)}^{y_2(x)}
                    \int_{z_1(x,y)}^{z_2(x,y)}$ f(x,y,z) dzdydx

                \hspace{3.9cm}
                = $\int_a^b \int_{z_1(x)}^{z_2(x)}
                    \int_{y_1(x,z)}^{y_2(x,z)}$ f(x,y,z) dydzdx

            \item D is a Type 2 elementary region
            
                \hspace{0.5cm}
                $\int \int \int_D$ f(x,y,z) dV
                = $\int_c^d \int_{x_1(y)}^{x_2(y)}
                    \int_{z_1(x,y)}^{z_2(x,y)}$ f(x,y,z) dzdxdy

                \hspace{3.9cm}
                = $\int_a^b \int_{z_1(y)}^{z_2(y)}
                    \int_{x_1(y,z)}^{x_2(y,z)}$ f(x,y,z) dxdzdy

            \item D is a Type 3 elementary region
            
                \hspace{0.5cm}
                $\int \int \int_D$ f(x,y,z) dV
                = $\int_e^f \int_{x_1(z)}^{x_2(z)}
                    \int_{y_1(x,z)}^{y_2(x,z)}$ f(x,y,z) dydxdz

                \hspace{3.9cm}
                = $\int_e^f \int_{y_1(z)}^{y_2(z)}
                    \int_{x_1(y,z)}^{x_2(y,z)}$ f(x,y,z) dxdydz
        \end{enumerate}
    \end{wtheorem}

    \begin{proof}
        Suppose D is a Type 1 elementary region.

        Let region R = [a,b] $\times$ [c,d] $\times$ [e,f] be such that
        c $\leq$ $y_1(x)$ $\leq$ y $\leq$ $y_2(x)$ $\leq$ d and
        
        e $\leq$ $z_1(x,y)$ $\leq$ z $\leq$ $z_2(x,y)$ $\leq$ f.

        Since $f^{ext}(x,y,z)$ = 0 for any
        (x,y,z) $\not \in$ D, then by {\color{red} theorem 4.1.5}:

        \hspace{0.5cm}
        $\int \int \int_D$ f(x,y,z) dV
        = $\int \int \int_R$ $f^{ext}(x,y,z)$ dV
        = $\int_a^b \int_c^d \int_e^f$ $f^{ext}(x,y,z)$ dzdydx
        
        \hspace{3.9cm}
        = $\int_a^b \int_c^d \int_{z_1(x,y)}^{z_2(x,y)}$ $f^{ext}(x,y,z)$ dzdydx

        \hspace{3.9cm}
        = $\int_a^b \int_{y_1(x)}^{y_2(x)} \int_{z_1(x,y)}^{z_2(x,y)}$
        f(x,y,z) dzdydx

        Proof is analogous for the other five cases.
    \end{proof}

    \vspace{0.5cm}



    \begin{example}
        Let f(x,y,z) = $x^2y + z$.
        Find $\int \int \int_D$ f(x,y,z) dV for

        (x,y,z) $\in$ TriangularPyramid((0,0,0),(3,0,0),(0,4,0),(3,4,5)).
    \end{example}

    \begin{tbox}
        $\int \int \int_D$ f(x,y,z) dV
        = $\int_0^3 \int_0^{\frac{4}{3}x} \int_0^{\frac{5}{4}y}$
            $x^2y + z$ dzdydx

        \hspace{3.3cm}
        = $\int_0^3 \int_0^{\frac{4}{3}x}$
            $\frac{5}{4}x^2y^2 + \frac{25}{32}y^2$ dydx
        = $\int_0^3$ $\frac{80}{81}x^5 + \frac{50}{81}x^3$ dx
        = $\frac{265}{2}$
    \end{tbox}
    
    \vspace{0.5cm}



    \begin{definition}{Volume}{14cm}
        The {\color{lblue} volume} of a region R $\subset$ $\mathbb{R}^3$:

        \hspace{0.5cm}
        V = $\int \int \int_R$ 1 dV
    \end{definition}

    \newpage





\subsection{ Change of Variables }

    \begin{wtheorem}{Linear Transformations Scaling Factor}{14cm}
        Let A $\in$ $M_{2 \times 2}(\mathbb{R})$ where det(A) $\not =$ 0
        such that T: $\mathbb{R}^2$ $\rightarrow$ $\mathbb{R}^2$:

        \hspace{0.5cm}
        (x,y) = T(u,v) = A
        $
        \begin{bmatrix}
            u \\
            v
        \end{bmatrix}
        $

        Transformation T is 1-1 and onto where if D is a parallelogram in the
        uv-plane, then T(D) is a parallelogram in the xy-plane where:

        \hspace{0.5cm}
        $\text{Vol}_2(\text{T(D)})$ = $|\text{det}(A)| * \text{Vol}_2(\text{D})$
    \end{wtheorem}

    \begin{proof}
        The proof that T is 1-1 and onto uses linear algebra,
        but it will be made simplier here.
        Note however if linear algebra is used in full effect,
        the given det(A) $\not =$ 0 satisfies the condition for 1-1 and onto.

        Let A =
        $
        \begin{bmatrix}
            a & b \\
            c & d
        \end{bmatrix}
        $.
        Suppose (a,c) = k(b,d) for some scalar k $\in$ $\mathbb{R}$.E

        \hspace{0.5cm}
        A $
        \begin{bmatrix}
            u \\
            v
        \end{bmatrix}
        $ =
        $
        \begin{bmatrix}
            au + bv \\
            cu + dv
        \end{bmatrix}
        $ =
        $
        \begin{bmatrix}
            a \\
            c
        \end{bmatrix}u
        +
        \begin{bmatrix}
            b \\
            d
        \end{bmatrix}v
        $ =
        $
        k
        \begin{bmatrix}
            b \\
            d
        \end{bmatrix}u
        +
        \begin{bmatrix}
            b \\
            d
        \end{bmatrix}v
        $ =
        $
        \begin{bmatrix}
            b \\
            d
        \end{bmatrix}(ku+v)
        $

        Let (u,v) = (1,-k) and (u,v) = (2,-2k).
        In both cases,
        A $
        \begin{bmatrix}
            u \\
            v
        \end{bmatrix}
        $ = 0
        so T is not 1-1 if (a,c) = k(b,d).
        But, if there is a k such that (a,c) = k(b,d),
        then det(A) = ad-bc = (kb)d-b(kd) = 0
        which contradicts that det(A) $\not =$ 0.
        Also, note if there is a k such that (a,c) = k(b,d),
        then A(u,v) consist of only multiples of (b,d)
        so T is not onto since vectors that are not multiples of
        (b,d) such as (b,d+1) cannot be reached.
        Now suppose there isn't a k such that (a,c) = k(b,d). Then A(u,v)
        cannot have two different $(u_1,v_1)$,$(u_2,v_2)$
        such that $A(u_1,v_1)$ = $A(u_2,v_2)$
        since 

        \hspace{0.5cm}
        $A(u_1,v_1)$ - $A(u_2,v_2)$ =
        (a,c)$(u_1-u_2)$ + (b,d)$(v_1-v_2)$ = 0

        so (a,c) = $-\frac{v_1 - v_2}{u_1 - u_2}$(b,d)
        and thus, there is a k = $-\frac{v_1 - v_2}{u_1 - u_2}$
        such that (a,c) = k(b,d) which is a contradiction.
        Thus, if there isn't a k such that (a,c) = k(b,d), then
        for two different $(u_1,v_1)$ and $(u_2,v_2)$, then 
        $A(u_1,v_1)$ $\not =$ $A(u_2,v_2)$ so T is 1-1.
        Also, note since there isn't a k such that (a,c) = k(b,d),
        then any (x,y) can be reached by solving the system of
        equations from A(u,v) = (x,y) for u,v and thus, T is onto.
        
        If there isn't a k such that (a,b) = k(c,d), then det(A) $\not =$ 0
        since if det(A) = ad-bc = 0, then ad=bc so $\frac{a}{b}$ = $\frac{c}{d}$.
        Note $\frac{a}{b}$ = $\frac{kc}{kd}$ = $\frac{c}{d}$ if there is a k
        such that (a,b) = k(c,d) which is a contradiction.
        Thus, by the restriction det(A) $\not =$ 0, then T is 1-1 and onto.

        \vspace{0.3cm}

        For any two $(u_1,v_1)$,$(u_2,v_2)$ where $(u_1,v_1)$ $\not  =$ k$(u_1,v_1)$
        for any k $\in$ $\mathbb{R}$, a parallelogram
        with sides $(u_1,v_1)$,$(u_2,v_2)$ can be formed in uv-plane.
        Then T maps $(u_1,v_1)$,$(u_2,v_2)$ into two different
        $A(u_1,v_1)$,$A(u_2,v_2)$ since T is 1-1 so $A(u_1,v_1)$,$A(u_2,v_2)$
        are sides of a parallelogram in the xy-plane. Thus:

        \hspace{0.5cm}
        $\text{Vol}_2(\text{T(D)})$
        = $|| A(u_1,v_1) \times A(u_2,v_2) ||$
        = $|| [(a,c)u_1 + (b,d)v_1] \times [(a,c)u_2 + (b,d)v_2] ||$
        
        \hspace{0.5cm}
        = $|| (a,c)u_1 \times (a,c)u_2
                + (a,c)u_1 \times (b,d)v_2
                + (b,d)v_1 \times (a,c)u_2
                + (b,d)v_1 \times (b,d)v_2 ||$

        \hspace{0.5cm}
        = $|| [u_1v_2 - v_1u_2] [(a,c) \times (b,d)] ||$
        = $|| (a,c) \times (b,d) ||$ $[u_1v_2 - v_1u_2]$

        \hspace{0.5cm}
        = $|ad-bd|$ $|| (u_1,v_1) \times (u_2,v_2) ||$
        = $|\text{det}(A)| * \text{Vol}_2(\text{D})$
    \end{proof}

    \newpage



    \begin{wtheorem}{Integral: Change of Variables}{14cm}
        The {\color{lblue} Jacobian} of T: $\mathbb{R}^2$
        $\rightarrow$ $\mathbb{R}^2$ where T(u,v) = (x,y):

        \hspace{0.5cm}
        $\frac{\partial(x,y)}{\partial(u,v)}$
        = det(DT(u,v))
        = det($
        \begin{bmatrix}
            \frac{\partial x}{\partial u} & \frac{\partial x}{\partial v} \\
            \frac{\partial y}{\partial u} & \frac{\partial y}{\partial v}
        \end{bmatrix}
        $)

        Let elementary region D be in uv-plane. If transformation T is
        $C^1$ (i.e. partial derivatives above are continuous),
        then for integrable f: T(D) $\rightarrow$ $\mathbb{R}$
        where
        
        x = x(u,v) and y = y(u,v):

        \hspace{0.5cm}
        $\int \int_\text{T(D)}$ f(x,y) dA(x,y)
        = $\int \int_\text{D}$ f(x(u,v),y(u,v))
            $|\frac{\partial(x,y)}{\partial(u,v)}|$ dA(u,v)

        \vspace{0.5cm}

        The Jacobian of T: $\mathbb{R}^3$
        $\rightarrow$ $\mathbb{R}^3$ where T(u,v,w) = (x,y,z):

        \hspace{0.5cm}
        $\frac{\partial(x,y,z)}{\partial(u,v,w)}$
        = det(DT(u,v))
        = det($
        \begin{bmatrix}
            \frac{\partial x}{\partial u}
            & \frac{\partial x}{\partial v}
            & \frac{\partial x}{\partial w} \\

            \frac{\partial y}{\partial u}
            & \frac{\partial y}{\partial v}
            & \frac{\partial y}{\partial w} \\

            \frac{\partial z}{\partial u}
            & \frac{\partial z}{\partial v}
            & \frac{\partial z}{\partial w} \\
        \end{bmatrix}
        $)

        Let elementary region D be in uvw-space. If transformation T is
        $C^1$, then for integrable f: T(D) $\rightarrow$ $\mathbb{R}$ where
        x = x(u,v,w) , y = y(u,v,w) , and z = z(u,v,w):

        \hspace{0.5cm}
        $\int \int \int_\text{T(D)}$ f(x,y,z) dV(x,y,z)
        
        \hspace{0.5cm}
        = $\int \int \int_\text{D}$ f(x(u,v,w),y(u,v,w),z(u,v,w))
            $|\frac{\partial(x,y,z)}{\partial(u,v,w)}|$ dV(u,v,w)
    \end{wtheorem}

    \begin{proof}
        Take the case for T: $\mathbb{R}^2$ $\rightarrow$ $\mathbb{R}^2$.
        The case for $\mathbb{R}^3$ is analogous.

        Since T is $C^1$, then DT(u,v) exist and since
        T(u,v) = A(u,v) = (x(u,v),y(u,v)), then:

        \hspace{0.5cm}
        DT(u,v) = A =
        $
        \begin{bmatrix}
            \frac{\partial x}{\partial u} & \frac{\partial x}{\partial v} \\
            \frac{\partial y}{\partial u} & \frac{\partial y}{\partial v} \\
        \end{bmatrix}
        $
        = $|\frac{\partial(x,y)}{\partial(u,v)}|$

        Thus, by {\color{red} theorem 4.3.1}:

        \hspace{0.5cm}
        $\int \int_\text{T(D)}$ f(x,y) dA(x,y)
        = $\underset{||P|| \rightarrow 0}{\lim}$
            $\sum_{i,j=1}^n f(x_{ij},y_{ij}) \Delta x_i \Delta y_j$

        \hspace{0.5cm}
        = $\underset{||P|| \rightarrow 0}{\lim}$
            $\sum_{i,j=1}^n f(x_{ij},y_{ij})
            \text{Vol}_2([\Delta x_i , \Delta y_j])$

        \hspace{0.5cm}
        = $\underset{||P|| \rightarrow 0}{\lim}$
            $\sum_{i,j=1}^n f(x_{ij}(u,v),y_{ij}(u,v))
            |\text{DT(u,v)}| \text{Vol}_2([\Delta u_i , \Delta v_j])$

        \hspace{0.5cm}
        = $\underset{||P|| \rightarrow 0}{\lim}$
            $\sum_{i,j=1}^n f(x_{ij}(u,v),y_{ij}(u,v))
            |\text{DT(u,v)}| \Delta u_i \Delta v_j$

        \hspace{0.5cm}
        = $\int \int_\text{D}$ f(x(u,v),y(u,v))
            $|\frac{\partial(x,y)}{\partial(u,v)}|$ dA(u,v)
    \end{proof}

    \vspace{0.5cm}



    \begin{example}
        Prove that the volume of a sphere with radius R is
        V = $\frac{4}{3}\pi R^3$.  
    \end{example}

    \begin{tbox}
        A sphere is $x^2 + y^2 + z^2$ = $R^2$.
        Instead of rectangular, use spherical coordinates.

        \hspace{0.5cm}
        $p^2$ = $r^2 + z^2$ = $x^2 + y^2 + z^2$ = $R^2$
        \hspace{1cm}
        $\Rightarrow$
        \hspace{1cm}
        p = R

        $\theta$ $\in$ $[0,2\pi]$ since a sphere in 2d is a circle
        and $\phi$ $\in$ $[0,\pi]$ since for (x,y) = (0,0), then
        z = [-R,R] where z = p cos($\phi$) = R cos($\phi$).

        Since x = p sin($\phi$) cos($\theta$) ,
        y = p sin($\phi$) sin($\theta$) ,
        and x = p cos($\phi$), then:

        \hspace{0.5cm}
        $|\frac{\partial(x,y,z)}{\partial(p,\theta,\phi)}|$
        = $|\begin{vmatrix}
            \scriptstyle \sin(\phi) \cos(\theta)
            & \scriptstyle -p \sin(\phi) \sin(\theta)
            & \scriptstyle p \cos(\phi) \cos(\theta) \\

            \scriptstyle \sin(\phi) \sin(\theta)
            & \scriptstyle p \sin(\phi) \cos(\theta)
            & \scriptstyle p \cos(\phi) \sin(\theta) \\

            \scriptstyle \cos(\phi)
            & \scriptstyle 0
            & \scriptstyle -p \sin(\phi)
        \end{vmatrix}|$
        = $|-p^2 \sin(\phi)|$
        = $p^2 \sin(\phi)$

        Volume
        = $\int \int \int_{x^2+y^2+z^2}$ 1 dV(x,y,z)
        = $\int_0^R \int_0^{\pi} \int_0^{2\pi}$
            $1 * p^2 \sin(\phi)$ $d\theta d\phi dp$

        \hspace{1.3cm}
        = $2\pi \int_0^R \int_0^{\pi}$
            $p^2 \sin(\phi)$ $d\phi dp$
        = $2\pi \int_0^R$ $2p^2$ $dp$ 
        = $\frac{4}{3} \pi R^3$
    \end{tbox}




