\newpage

\section[Day 6: Surface Integrals]{ Surface Integrals }

\subsection{ Parametrized Surfaces }

    \begin{definition}{ Parametrized Surfaces }{14cm}
        A {\color{lblue} parametrized surface}
        is a continuous function X: D $\subset$ $\mathbb{R}^2$
        $\rightarrow$ $\mathbb{R}^3$.

        \vspace{0.5cm}

        Let surface S be parametrized by
        differentiable X(s,t) = (x(s,t), y(s,t), z(s,t)).

        Then the tangent vectors:

        \hspace{0.5cm}
        $T_s(s,t)$
        = $\frac{\partial X}{\partial s}(s,t)$
        = $(\frac{\partial x}{\partial s}(s,t),
            \frac{\partial y}{\partial s}(s,t),
            \frac{\partial z}{\partial s}(s,t))$

        \hspace{0.5cm}
        $T_t(s,t)$
        = $\frac{\partial X}{\partial t}(s,t)$
        = $(\frac{\partial x}{\partial t}(s,t),
            \frac{\partial y}{\partial t}(s,t),
            \frac{\partial z}{\partial t}(s,t))$

        S is {\color{lblue} smooth} if X is $C^1$ and the
        {\color{lblue} standard normal vector} of X:

        \hspace{0.5cm}
        $N_X(s,t)$ = $T_s(s,t)$ $\times$ $T_t(s,t)$ $\not =$ 0.

        S is {\color{lblue} piecewise smooth} if
        X = \{$X_1,...,X_m$\} where each parametrized surface $X_i$
        is $C^1$ and $N_{X_i}(s,t)$ $\not =$ 0
    \end{definition}

    \vspace{0.5cm}




    \begin{wtheorem}{Surface Area}{14cm}
        Let surface S be parametrized by $C^1$ function X:
        D $\subset$ $\mathbb{R}^2$ $\rightarrow$ $\mathbb{R}^3$.
        
        Then the surface area of S:

        \hspace{0.5cm}
        $\int \int_D$ $|| N_X(s,t) ||$ dA
        = $\int \int_D$ $\sqrt{(\frac{\partial(x,y)}{\partial(s,t)})^2
                                + (\frac{\partial(x,z)}{\partial(s,t)})^2
                                + (\frac{\partial(y,z)}{\partial(s,t)})^2}$ dA

        If the surface S is z = f(x,y), then the surface area of S:

        \hspace{0.5cm}
        $\int \int_D$ $\sqrt{f_x^2 + f_y^2 + 1}$ dA
    \end{wtheorem}

    \vspace{0.5cm}





\subsection{ Surface Integrals }

    \begin{wtheorem}{Scalar Surface Integrals}{14cm}
        Let surface S be parametrized by smooth X(s,t):
        D $\subset$ $\mathbb{R}^2$ $\rightarrow$ $\mathbb{R}^3$ and
        
        f(x,y,z): S $\subset$ R $\subset$ $\mathbb{R}^3$
        $\rightarrow$ $\mathbb{R}$ be continuous.

        Then the {\color{lblue} scalar surface integral} of f along X(s,t):

        \hspace{0.5cm}
        $\int \int_{X(s,t)}$ f dS
        = $\int \int_D$ f(X(s,t)) $||N_X(s,t)||$ dA
    \end{wtheorem}

    \vspace{0.5cm}



    \begin{wtheorem}{Vector Surface Integrals}{14cm}
        Let surface S be parametrized by smooth X(s,t):
        D $\subset$ $\mathbb{R}^2$ $\rightarrow$ $\mathbb{R}^3$ and
        vector field F(x,y,z): S $\subset$ R $\subset$ $\mathbb{R}^3$
        $\rightarrow$ $\mathbb{R}^3$ be continuous.

        Then the {\color{lblue} vector surface integral} of F along X(s,t):

        \hspace{0.5cm}
        $\int \int_{X(s,t)}$ F $\cdot$ dS
        = $\int \int_D$ F(X(s,t)) $\cdot$ $N_X(s,t)$ dA
        
        The vector surface integral is also called the
        {\color{lblue} flux of F across S = X(s,t)}.
    \end{wtheorem}

    \newpage





\subsection{ Stokes's \& Gauss's Theorems }

    \begin{definition}{Induced Orientation}{14cm}
        Let S be a bounded, piecewise smooth surface in $\mathbb{R}^3$
        with boundary defined by a simple, closed curve C.
        Let unit normal vector n be orthogonal to S at a point
        so surface S has an orientation of n.

        Then for points on S, let there be a spinning disc with a direction
        such that the curl is in the same direction as n at such points.

        Then C has an {\color{lblue} induced orientation} from S
        if the orientation of curve C is in the same direction as
        the spinning discs of points near C.
    \end{definition}

    \vspace{0.5cm}



    \begin{wtheorem}{Stokes's Theorem}{14cm}
        Let S be a bounded, piecewise smooth, oriented surface in $\mathbb{R}^3$
        with boundary defined by finitely many simple, closed, piecewise $C^1$
        curves C = \{$C_1,...,C_m$\} which have an induced orientation from S.

        Then for $C^1$ vector field F(x,y,z): S $\subset$ R $\subset$
        $\mathbb{R}^3$ $\rightarrow$ $\mathbb{R}^3$:

        \hspace{0.5cm}
        $\int \int_S$ $\nabla \times F$ $\cdot$ dS
        = $\oint_C$ F $\cdot$ ds
    \end{wtheorem}

    \vspace{0.5cm}



    \begin{wtheorem}{Gauss's Theorem: Divergence Theorem in $\mathbb{R}^3$}{14cm}
        Let D be a bounded region in $\mathbb{R}^3$ with boundary defined
        by finitely many closed, piecewise smooth surfaces S = \{$S_1,...,S_m$\}
        oriented with unit normal vectors that points away from D.

        Then for $C^1$ vector field F(x,y,z): D $\subset$ R $\subset$
        $\mathbb{R}^3$ $\rightarrow$ $\mathbb{R}^3$:

        \hspace{0.5cm}
        $\oiint_S$ F $\cdot$ dS
        = $\int \int \int_D$ $\nabla \cdot F$ dV
    \end{wtheorem}






























