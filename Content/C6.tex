\newpage

\section[Day 6: Surface Integrals]{ Surface Integrals }

\subsection{ Parametrized Surfaces }

    \begin{definition}{ Parametrized Surfaces }{14cm}
        A {\color{lblue} parametrized surface}
        is a continuous function X: D $\subset$ $\mathbb{R}^2$
        $\rightarrow$ $\mathbb{R}^3$.

        \vspace{0.5cm}

        Let surface S be parametrized by
        differentiable X(s,t) = (x(s,t), y(s,t), z(s,t)).

        Then the tangent vectors:

        \hspace{0.5cm}
        $T_s(s,t)$
        = $\frac{\partial X}{\partial s}(s,t)$
        = $(\frac{\partial x}{\partial s}(s,t),
            \frac{\partial y}{\partial s}(s,t),
            \frac{\partial z}{\partial s}(s,t))$

        \hspace{0.5cm}
        $T_t(s,t)$
        = $\frac{\partial X}{\partial t}(s,t)$
        = $(\frac{\partial x}{\partial t}(s,t),
            \frac{\partial y}{\partial t}(s,t),
            \frac{\partial z}{\partial t}(s,t))$

        S is {\color{lblue} smooth} if X is $C^1$ and the
        {\color{lblue} standard normal vector} of X:

        \hspace{0.5cm}
        $N_X(s,t)$ = $T_s(s,t)$ $\times$ $T_t(s,t)$ $\not =$ 0.

        S is {\color{lblue} piecewise smooth} if
        X = \{$X_1,...,X_m$\} where each parametrized surface $X_i$
        is $C^1$ and $N_{X_i}(s,t)$ $\not =$ 0
    \end{definition}

    \vspace{0.5cm}




    \begin{wtheorem}{Surface Area}{14cm}
        Let surface S be parametrized by $C^1$ function X:
        D $\subset$ $\mathbb{R}^2$ $\rightarrow$ $\mathbb{R}^3$.
        
        Then the surface area of S:

        \hspace{0.5cm}
        $\int \int_D$ $|| N_X(s,t) ||$ dA
        = $\int \int_D$ $\sqrt{(\frac{\partial(x,y)}{\partial(s,t)})^2
                                + (\frac{\partial(x,z)}{\partial(s,t)})^2
                                + (\frac{\partial(y,z)}{\partial(s,t)})^2}$ dA

        If the surface S is z = f(x,y), then the surface area of S:

        \hspace{0.5cm}
        $\int \int_D$ $\sqrt{f_x^2 + f_y^2 + 1}$ dA
    \end{wtheorem}

    \begin{proof}
        Let P = \{$s_0,...,s_n$\} $\times$ \{$t_0,...,t_n$\}
        be a partition of region D.
        For each $\Delta s_i$ $\times$ $\Delta t_j$, choose
        a point $(s_i^*,t_j^*)$ $\in$ $[s_{i-1},s_i] \times [t_{j-1},t_j]$.
        Let S be the set of all points $(s_i^*,t_j^*)$.

        Then for each $\Delta s_i$ $\times$ $\Delta t_j$,
        the $||T_s(s_i^*,t_j^*)\Delta s_i \times T_t(s_i^*,t_j^*)\Delta t_i||$
        is an approximate area for the surface at
        $\Delta s_i$ $\times$ $\Delta t_j$ $\in$ D.
        Thus:

        \hspace{0.5cm}
        Surface Area
        = $\underset{||P|| \rightarrow 0}{\lim}$ $\sum_{i,j=1}^n$
            $||T_s(s_i^*,t_j^*)\Delta s_i \times T_t(s_i^*,t_j^*)\Delta t_i||$

        \hspace{2.9cm}
        = $\underset{||P|| \rightarrow 0}{\lim}$ $\sum_{i,j=1}^n$
            $||T_s(s_i^*,t_j^*) \times T_t(s_i^*,t_j^*)||\Delta s_i\Delta t_i$
        = $\int \int_D$ $|| N_X(s,t) ||$ dA
    \end{proof}

    \vspace{0.5cm}



    \begin{example}
        Prove that the surface area of a sphere of radius R is $4\pi R^2$.
    \end{example}

    \begin{tbox}
        Using spherical coordinates,
        x = $R \sin(\phi) \cos(\theta)$ , y = $R \sin(\phi) \sin(\theta)$ ,
        and z = $R \cos(\phi)$ for $\phi$ $\in$ $[0,\pi]$
        and $\theta$ $\in$ $[0,2\pi]$. Let X($\theta,\phi$) be
        a parametrization of the sphere. Then:

        \hspace{0.5cm}
        $N_X(\theta,\phi)$
        = $\begin{vmatrix}
            e_1 & e_2 & e_3 \\
            
            -R \sin(\phi) \sin(\theta)
            & R \sin(\phi) \cos(\theta)
            & 0 \\

            R \cos(\phi) \cos(\theta)
            & R \cos(\phi) \sin(\theta)
            & -R \sin(\phi) \\
        \end{vmatrix}$
        
        \hspace{2.2cm}
        = $(-R^2 \sin^2(\phi) \cos(\theta), -R^2 \sin^2(\phi) \sin(\theta) ,
            -R^2 \sin(\phi) \cos(\phi))$

        
        $\int_0^{2\pi} \int_0^{\pi}$
            $|| (-R^2 \sin^2(\phi) \cos(\theta), -R^2 \sin^2(\phi) \sin(\theta) ,
            -R^2 \sin(\phi) \cos(\phi)) ||$ d$\phi$d$\theta$
    
        = $\int_0^{2\pi} \int_0^{\pi}$ $R^2 \sin(\phi)$ d$\phi$d$\theta$
        = $\int_0^{2\pi}$ $2R^2$ d$\theta$
        = $4\pi R^2$
    \end{tbox}

    \newpage





\subsection{ Surface Integrals }

    \begin{definition}{Scalar Surface Integrals}{14cm}
        Let surface S be parametrized by smooth X(s,t):
        D $\subset$ $\mathbb{R}^2$ $\rightarrow$ $\mathbb{R}^3$ and
        
        f(x,y,z): S $\subset$ R $\subset$ $\mathbb{R}^3$
        $\rightarrow$ $\mathbb{R}$ be continuous.

        Then the {\color{lblue} scalar surface integral} of f along X(s,t):

        \hspace{0.5cm}
        $\int \int_{X(s,t)}$ f dS
        = $\int \int_D$ f(X(s,t)) $||N_X(s,t)||$ dA
    \end{definition}

    \vspace{0.5cm}



    \begin{example}
        Let f(x,y,z) = 6xy. Find the scalar surface integral of f along
        the surface x+y+z=1 in the first octant.
    \end{example}

    \begin{tbox}
        Let X(s,t) = (s,t,1-s-t) for s $\in$ [0,1] and t $\in$ [0,1-s].

        \hspace{0.5cm}
        $X_s(s,t)$ = (1,0,-1)
        \hspace{1cm}
        $X_t(s,t)$ = (0,1,-1)
        \hspace{1cm}
        $N_X(s,t)$ = (1,1,1)

        $\int \int_{X(s,t)}$ f dS
        = $\int_0^1 \int_0^{1-s}$ 6st $\sqrt{1^2 + 1^2 + 1^2}$ dtds
        = $3\sqrt{3} \int_0^1$ $s - 2s^2 + s^3$ ds
        = $\frac{\sqrt{3}}{4}$
    \end{tbox}

    \vspace{0.5cm}



    \begin{definition}{Vector Surface Integrals}{14cm}
        Let surface S be parametrized by smooth X(s,t):
        D $\subset$ $\mathbb{R}^2$ $\rightarrow$ $\mathbb{R}^3$ and
        vector field F(x,y,z): S $\subset$ R $\subset$ $\mathbb{R}^3$
        $\rightarrow$ $\mathbb{R}^3$ be continuous.

        Then the {\color{lblue} vector surface integral} of F along X(s,t):

        \hspace{0.5cm}
        $\int \int_{X(s,t)}$ F $\cdot$ dS
        = $\int \int_D$ F(X(s,t)) $\cdot$ $N_X(s,t)$ dA

        \vspace{0.3cm}

        {\color{red} Physical Understanding: Vector Surface Integrals
        = Divergence of Fluid}

        \hspace{0.5cm}
        \begin{minipage}{13.5cm}
            Let F measure the flow of water.

            \hspace{0.5cm}
            $\int \int_D$ F(X(s,t)) $\cdot$ $N_X(s,t)$ dA
            = $\int \int_D$
                F(X(s,t)) $\cdot$
                $\frac{N_X(s,t)}{||N_X(s,t)||}$ $||N_X(s,t)||$ dA

            \begin{itemize}[leftmargin=0.5cm, itemsep=0.01cm]
                \item F(X(s,t)) $\cdot$ $\frac{N_X(s,t)}{||N_X(s,t)||}$
                    = $||F(X(s,t))||$
                        $\scriptstyle \frac{||N_X(s,t)||}{||N_X(s,t)||}$
                        $\cos(\theta)$
                    = $||F(X(s,t))||$ $\cos(\theta)$
                    is the component of F that moves in direction of $N_x(s,t)$
                    which is orthogonal to F(X(s,t)).

                \item $||N_X(s,t)||$ dA is the change in surface area
            \end{itemize}

            Thus, $\int \int_D$
                F(X(s,t)) $\cdot$
                $\frac{N_X(s,t)}{||N_X(s,t)||}$ $||N_X(s,t)||$ dA
            is the total change in fluid across the surface X(s,t).
            Thus, the vector surface integral is also called the
        {\color{lblue} flux of F across S = X(s,t)}.
        \end{minipage}
    \end{definition}

    \vspace{0.5cm}



    \begin{example}
        Let F(x,y,z) = $(x^3,-y^3,0)$ and surface S be bounded by
        $x^2+y^2$ = 4 from z $\in$ [0,4] with circles at z=0,z=4.
        Let the normal vectors be pointing outwards.

        Find the vector surface integral of F along surface S.
    \end{example}

    \begin{tbox}
        Surface z consist of 3 parts:
        
        \hspace{0.5cm}
        $S_1$: $x^2+y^2=4$, z=0
        \hspace{0.5cm}
        $S_2$: $x^2+y^2=4$, z=4
        \hspace{0.5cm}
        $S_3$: $x^2+y^2=4$, z=[0,4]

        Since the outward unit normal vector for $S_1$ = (0,0,1)
        and $S_2$ = (0,0,-1), then:

        \hspace{0.5cm}
        $\int \int_S$ F $\cdot$ dS
        = $\iint_{D_1}$ F($X_{S_1}$)$\cdot$$n_1$ dA
            + $\iint_{D_2}$ F($X_{S_2}$)$\cdot$$n_2$ dA
            + $\iint_{D_3}$ F($X_{S_3}$)$\cdot$$n_3$ dA
        
        \hspace{0.5cm}
        = 0 + 0 + $\int \int_{D_3}$ F($X_{S_3}$) $\cdot$ $n_3$ dA
        = $\int \int_{D_3}$ F($X_{S_3}$) $\cdot$ $n_3$ dA

        $S_3$ can be parametrized X(s,t) = $(2\cos(s),2\sin(s),t)$
        where s $\in$ $[0,2\pi]$, t $\in$ [0,4].

        \hspace{0.5cm}
        $N_X(s,t)$
        = $(-2\sin(s),2\cos(s),0) \times (0,0,1)$
        = $(2\cos(s),2\sin(s),0)$

        which has the outward orientation so $N_X(s,t)$ is correct.

        \hspace{0.5cm}
        $\int \int_{D_3}$ F($X_{S_3}$) $\cdot$ $n_3$ dA
        = $\int_0^{2\pi} \int_0^4$ $16\cos^4(s) - 16\sin^4(s)$ dtds

        \hspace{4.5cm}
        = $64 \int_0^{2\pi}$ $\cos(2s)$ ds = 0
    \end{tbox}

    \newpage



    \begin{definition}{Reparametrization for Surfaces}{14cm}
        Let $X_1$: $D_1$ $\subset$ $\mathbb{R}^2$
        $\rightarrow$ $\mathbb{R}^3$ be a smooth parametrization of surface S.

        Then smooth $X_2$: $D_2$ $\subset$ $\mathbb{R}^2$
        $\rightarrow$ $\mathbb{R}^3$ is a {\color{lblue} reparametrization}
        of $X_1$ if there is a 1-1 and onto v: $D_2$ $\rightarrow$ $D_1$ such that:

        \hspace{0.5cm}
        $X_2(s,t)$ = $X_1(v(s,t))$

        \vspace{0.3cm}

        Let $X_1(v_1,v_2)$ = $(x_1,y_1,z_1)$, $X_2(s,t)$ = $(x_2,y_2,z_2)$, and
        $v(s,t)$ = $(v_1,v_2)$. Then:

        \hspace{0.5cm}
        $\begin{bmatrix}
            \frac{\partial x_2}{\partial s} & \frac{\partial x_2}{\partial t} \\
            \frac{\partial y_2}{\partial s} & \frac{\partial y_2}{\partial t} \\
            \frac{\partial z_2}{\partial s} & \frac{\partial z_2}{\partial t}
        \end{bmatrix}$
        =
        $\begin{bmatrix}
            \frac{\partial x_1}{\partial v_1} & \frac{\partial x_1}{\partial v_2} \\
            \frac{\partial y_1}{\partial v_1} & \frac{\partial y_1}{\partial v_2} \\
            \frac{\partial z_1}{\partial v_1} & \frac{\partial z_1}{\partial v_2}
        \end{bmatrix}
        \begin{bmatrix}
            \frac{\partial v_1}{\partial s} & \frac{\partial v_1}{\partial t} \\
            \frac{\partial v_2}{\partial s} & \frac{\partial v_2}{\partial t}
        \end{bmatrix}$
        =
        $\scriptstyle \begin{bmatrix}
            \frac{\partial x_1}{\partial v_1}\frac{\partial v_1}{\partial s}
                + \frac{\partial x_1}{\partial v_2}\frac{\partial v_2}{\partial s}
            & \frac{\partial x_1}{\partial v_1}\frac{\partial v_1}{\partial t}
                + \frac{\partial x_1}{\partial v_2}\frac{\partial v_2}{\partial t} \\
            
            \frac{\partial y_1}{\partial v_1}\frac{\partial v_1}{\partial s}
                + \frac{\partial y_1}{\partial v_2}\frac{\partial v_2}{\partial s}
            & \frac{\partial y_1}{\partial v_1}\frac{\partial v_1}{\partial t}
                + \frac{\partial y_1}{\partial v_2}\frac{\partial v_2}{\partial t} \\

            \frac{\partial z_1}{\partial v_1}\frac{\partial v_1}{\partial s}
                + \frac{\partial z_1}{\partial v_2}\frac{\partial v_2}{\partial s}
            & \frac{\partial z_1}{\partial v_1}\frac{\partial v_1}{\partial t}
                + \frac{\partial z_1}{\partial v_2}\frac{\partial v_2}{\partial t} \\
        \end{bmatrix}$

        \hspace{0.5cm}
        $X_{2s}$
        = $X_{1v_1}\frac{\partial v_1}{\partial s}$
            + $X_{1v_2}\frac{\partial v_2}{\partial s}$
        \hspace{2cm}
        $X_{2t}$
        = $X_{1v_1}\frac{\partial v_1}{\partial t}$
            + $X_{1v_2}\frac{\partial v_2}{\partial t}$

        \hspace{0.5cm}
        $N_{X_2}$
        = $X_{2s}$ $\times$ $X_{2t}$
        = $[X_{1v_1} \times X_{1v_2}]
            (\frac{\partial v_1}{\partial s}\frac{\partial v_2}{\partial t}
            - \frac{\partial v_2}{\partial s}\frac{\partial v_1}{\partial t})$
        = $N_{X_1} \frac{\partial(v_1,v_2)}{\partial(s,t)}$

        Thus:

        \hspace{0.5cm}
        {\color{lblue} Orientation preserving}:
        \hspace{0.5cm}
        $\frac{\partial(v_1,v_2)}{\partial(s,t)}$ $>$ 0

        \hspace{0.5cm}
        {\color{lblue} Orientation reversing}:
        \hspace{0.75cm}
        $\frac{\partial(v_1,v_2)}{\partial(s,t)}$ $<$ 0
    \end{definition}

    \vspace{0.5cm}



    \begin{wtheorem}{Scalar Surface Integrals are unaffected
    by Reparametrization}{14cm}
        Let surface S be parametrized by smooth $X_1(v_1,v_2)$: $D_1$ $\subset$
        $\mathbb{R}^2$ $\rightarrow$ $\mathbb{R}^3$
        and f(x,y,z): S $\subset$ R $\subset$ $\mathbb{R}^3$
        $\rightarrow$ $\mathbb{R}^3$ be continuous.

        Let $X_2(s,t)$: $D_2$ $\rightarrow$ $\mathbb{R}^3$
        be a reparametrization of $X_1(s,t)$. Then:

        \hspace{0.5cm}
        $\int \int_{X_2(s,t)}$ f dS
        = $\int \int_{X_1(s,t)}$ f dS
    \end{wtheorem}

    \begin{proof}
        $\int \int_{X_2(s,t)}$ f dS
        = $\int \int_{D_2}$ f($X_2(s,t)$) $||N_{X_2(s,t)}||$ dA

        \hspace{2.4cm}
        = $\int \int_{D_2}$ f($X_1(v(s,t))$)
            $||N_{X_1(v_1(s,t),v_2(s,t))}
            \frac{\partial(v_1,v_2)}{\partial(s,t)}||$ dA
        
        \hspace{2.4cm}
        = $\int \int_{D_2}$ f($X_1(v(s,t))$)
            $||N_{X_1(v_1(s,t),v_2(s,t))}||$
            $|\frac{\partial(v_1,v_2)}{\partial(s,t)}|$ dA

        Then by {\color{red} theorem 4.3.2}:

        \hspace{2.4cm}
        = $\int \int_{D_1}$ f($X_1(v_1,v_2)$) $||N_{X_1(v_1,v_2)}||$ dA
        = $\int \int_{X_1(s,t)}$ f dS
    \end{proof}

    \vspace{0.5cm}



    \begin{wtheorem}{Vector Surface Integrals are affected
    by Reparametrization}{14cm}
        Let surface S be parametrized by smooth $X_1(v_1,v_2)$: $D_1$ $\subset$
        $\mathbb{R}^2$ $\rightarrow$ $\mathbb{R}^3$
        and vector field F(x,y,z): S $\subset$ R $\subset$ $\mathbb{R}^3$
        $\rightarrow$ $\mathbb{R}^3$ be continuous.

        Let $X_2(s,t)$: $D_2$ $\rightarrow$ $\mathbb{R}^3$
        be a reparametrization of $X_1(s,t)$. Then:

        \begin{itemize}[itemsep=0.1cm]
            \item If $X_2(s,t)$ is orientation preserving:
            
                \hspace{0.5cm}
                $\int \int_{X_2(s,t)}$ F $\cdot$ dS
                = $\int \int_{X_1(s,t)}$ F $\cdot$ dS
            
            \item If $X_2(s,t)$ is orientation reversing:
                
                \hspace{0.5cm}
                $\int \int_{X_2(s,t)}$ F $\cdot$ dS
                = $-\int \int_{X_1(s,t)}$ F $\cdot$ dS
        \end{itemize}
    \end{wtheorem}

    \begin{proof}
        $\int \int_{X_2(s,t)}$ F $\cdot$ dS
        = $\int \int_{D_2}$ F($X_2(s,t)$) $\cdot$ $N_{X_2(s,t)}$ dA

        \hspace{2.75cm}
        = $\int \int_{D_2}$ F($X_1(v(s,t))$)
                            $\cdot$ $N_{X_1(v_1(s,t),v_2(s,t))}$
                            $\frac{\partial(v_1,v_2)}{\partial(s,t)}$ dA

        If $X_2(s,t)$ is orientation preserving,
        then by {\color{red} theorem 4.3.2}:

        \hspace{2.75cm}
        = $\int \int_{D_1}$ F($X_1(v_1,v_2)$)
                            $\cdot$ $N_{X_1(v_1,v_2)}$ dA
        = $\int \int_{X_1(s,t)}$ F $\cdot$ dS

        If $X_2(s,t)$ is orientation preserving,
        then by {\color{red} theorem 4.3.2}:

        \hspace{2.75cm}
        = $-\int \int_{D_1}$ F($X_1(v_1,v_2)$)
                            $\cdot$ $N_{X_1(v_1,v_2)}$ dA
        = $-\int \int_{X_1(s,t)}$ F $\cdot$ dS
    \end{proof}

    \newpage





\subsection{ Stokes's \& Gauss's Theorems }

    \begin{definition}{Induced Orientation}{14cm}
        Let S be a bounded, piecewise smooth surface in $\mathbb{R}^3$
        with boundary defined by a simple, closed curve C.
        Let unit normal vector n be orthogonal to S at a point
        so surface S has an orientation of n.

        Then for points on S, let there be a spinning disc with a direction
        such that the curl is in the same direction as n at such points.

        Then C has an {\color{lblue} induced orientation} from S
        if the orientation of curve C is in the same direction as
        the spinning discs of points near C.
    \end{definition}

    \vspace{0.5cm}



    \begin{wtheorem}{Stokes's Theorem}{14cm}
        Let S be a bounded, piecewise smooth, oriented surface in $\mathbb{R}^3$
        with boundary defined by finitely many simple, closed, piecewise $C^1$
        curves C = \{$C_1,...,C_m$\} which have an induced orientation from S.

        Then for $C^1$ vector field F(x,y,z): S $\subset$ R $\subset$
        $\mathbb{R}^3$ $\rightarrow$ $\mathbb{R}^3$:

        \hspace{0.5cm}
        $\int \int_S$ $\nabla \times F$ $\cdot$ dS
        = $\oint_C$ F $\cdot$ ds
    \end{wtheorem}

    \vspace{0.5cm}



    \begin{example}
        Let F(x,y) = (x+y,x-y)
        and S be the hemisphere $x^2+y^2+z^2$ = 1 for z $\geq$ 0.

        Prove $\int \int_S$ $\nabla \times F$ $\cdot$ dS = 0.

        Recall the example at 5.5.4 which used the example from 5.3.2.
    \end{example}

    \begin{tbox}
        Suppose S has outward normal vectors. The case for inward normal vectors
        is analogous.
        Since S is a bounded, piecewise smooth, oriented surface with a
        boundary C: $x^2+y^2$ = 1 with an induced orientation from S, then:

        \hspace{0.5cm}
        $\int \int_S$ $\nabla \times F$ $\cdot$ dS
        = $\oint_C$ F $\cdot$ ds

        Recall the example at 5.5.4 and 5.3.2
        calculated that the $\oint_C$ F $\cdot$ ds = 0.

        Thus, $\int \int_S$ $\nabla \times F$ $\cdot$ dS = 0.

        Note that region S is irrelevant, only
        its boundary is important.

        Stokes' Theorem shows that given a $C^1$ vector field,
        the vector surface integral of the curl of any surface
        is equivalent to the vector surface integral of the curl
        of another surface as long as the surfaces have the same boundary.
    \end{tbox}

    \newpage



    \begin{wtheorem}{Gauss's Theorem: Divergence Theorem in $\mathbb{R}^3$}{14cm}
        Let D be a bounded region in $\mathbb{R}^3$ with boundary defined
        by finitely many closed, piecewise smooth surfaces S = \{$S_1,...,S_m$\}
        oriented with unit normal vectors that points away from D.

        Then for $C^1$ vector field F(x,y,z): D $\subset$ R $\subset$
        $\mathbb{R}^3$ $\rightarrow$ $\mathbb{R}^3$:

        \hspace{0.5cm}
        $\oiint_S$ F $\cdot$ dS
        = $\int \int \int_D$ $\nabla \cdot F$ dV
    \end{wtheorem}

    \vspace{0.5cm}



    \begin{example}
        Recall the example at 6.2.2.

        Let F(x,y,z) = $(x^3,-y^3,0)$ and surface S be bounded by
        $x^2+y^2$ = 4 from z $\in$ [0,4] with circles at z=0,z=4.
        Let the normal vectors be pointing outwards.

        Justify why $\oiint_S$ F $\cdot$ dS = 0. 
    \end{example}

    \begin{tbox}
        Since S is a boundary for bounded region D:
        $x^2+y^2$ $\leq$ 4 from z $\in$ [0,4]
        in which S is oriented with outward normal vectors, then:

        \hspace{0.5cm}
        $\oiint_S$ F $\cdot$ dS
        = $\int \int \int_D$ $\nabla \cdot F$ dV
        
        Region D: $x^2 + y^2 \leq 4$ for z $\in$ [0,4] be
        be redefined in cylindrical coordinates such that
        r $\in$ [0,2], $\theta$ $\in$ $[0,2\pi]$, and z $\in$ [0,4].

        \hspace{0.5cm}
        $\nabla \cdot F$
        = $\frac{\partial}{\partial x}x^3
            + \frac{\partial}{\partial y}(-y^3)
            + \frac{\partial}{\partial z}0$
        = $3x^2 - 3y^2$
        
        Applying {\color{red} theorem 4.3.2}
        for x = r cos($\theta$), y = r sin($\theta$), and z = z:

        \hspace{0.5cm}
        $\frac{\partial(x,y,z)}{\partial(r,\theta,z)}$
        = $\begin{vmatrix}
            \cos(\theta) & -r \sin(\theta) & 0 \\
            \sin(\theta) & r \cos(\theta) & 0 \\
            0 & 0 & 1
        \end{vmatrix}$
        = r

        \hspace{0.5cm}
        $\int \int \int_D$ $\nabla \cdot F$ dV
        = $\int \int \int_{x^2 + y^2 \leq 4, z \in [0,4]}$ $\nabla \cdot F$ dV

        \hspace{3.6cm}
        = $\int_0^4 \int_0^2 \int_0^{2\pi}$
            $[3\cos^2(\theta) - 3\sin^2(\theta)]$ r d$\theta$drdz

        \hspace{3.6cm}
        = $\int_0^4 \int_0^2 \int_0^{2\pi}$ $3r \cos(2\theta)$ d$\theta$drdz
        = $\int_0^4 \int_0^2$ 0 drdz
        = 0

        which is exactly the value calculated in the previous example.
    \end{tbox}

    \vspace{0.5cm}






























